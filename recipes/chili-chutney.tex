\begin{recipe}{Chili-Chutney} {1 Glas} {2h \faIcon[regular]{clock}, 30min \faIcon[regular]{user}}

  \freeform{}\textit{Fruchtig scharfes Chutney.}

  \ing[30]{}{rote Chilis (mild bis mittelscharf)}

  \textbf{10min}
  Chilis waschen, die Kerne entfernen und das Fruchtfleisch in kleine Würfel schneiden.

  \ing[230]{ml}{Tafelessig}

  \textbf{2h}
  Würfel mit dem Tafelessig mischen und darin 2 Stunden einweichen.

  \ing[1]{Stück}{frischer Ingwer 1–2cm}
  \ing[1]{TL}{Öl}

  \textbf{10min}
  Ingwer schälen und in feine Würfel hacken.
  Öl in einer Pfanne erhitzen und den Ingwer darin 3 Minuten dünsten.
  Ingwer zu den Chilis geben.

  \ing[120]{g}{Zucker}
  \ing[5]{EL}{Wasser}
  \ing[1]{TL}{Kreuzkümmel}
  \ing[1]{Prise}{Salz}

  \textbf{5min}
  Zucker mit dem Wasser in eine Pfanne geben und den Zucker unter Rühren im Wasser auflösen.
  Den Tafelessig abgießen.
  Dann die Chilis, Kreuzkümmel und Salz hinzugeben und bei mittlerer Hitze 5 Minuten köcheln lassen, bis das Chutney eindickt.

  \newstep{}\textbf{5min}
  Chutney entweder sofort in ein sterilisiertes Glas füllen oder etwas abkühlen lassen und direkt genießen.

  \freeform{}\hrulefill{}

  \freeform{}\faIcon[regular]{lightbulb}
  Einmachglas sterilisieren.

\end{recipe}
