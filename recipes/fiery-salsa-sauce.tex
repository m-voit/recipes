\begin{recipe}{Feurige-Salsa-Sauce} {500ml} {30 Minuten Arbeitszeit, 30 Minuten Wartezeit, 60 Minuten Gesamt}

  \freeform{}\textit{Ergibt eine feurige Sauce.}

  \ingredient[1]{x}{Zwiebel}
  \ingredient[2]{x}{Knoblauchzehe}
  \ingredient[2]{x}{Chilischoten (Jalapenos)}
  \ingredient[1]{EL}{Olivenöl}

  Zuerst Zwiebel und Knoblauchzehen schälen und fein würfeln.
  Kerne der Chilischoten entfernen und Schoten würfeln.
  Olivenöl in einer beschichteten Pfanne erhitzen und die Zwiebeln, sowie den Knoblauch glasig dünsten.

  \ingredient[425]{ml}{Tomaten, stückig}
  \ingredient[4]{EL}{Apfelessig}
  \ingredient[2]{EL}{Rohrzucker braun}

  Anschließend Chilischoten, stückige Tomaten, Apfelessig und Rohrzucker in die Pfanne geben.

  \ingredient[]{}{Pfeffer schwarz}
  \ingredient[]{}{Meersalz}

  Die Sauce ca. 30 Minuten bei schwacher Hitze köcheln lassen.
  Gelegentlich umrühren und mit Salz und Pfeffer, sowie Apfelessig abschmecken.
  Heiß in eine saubere Flasche füllen und anschließend über Nacht ziehen lassen.

  \freeform{}\hrulefill{}

  \freeform{}\textbf{Hitzestufe:}
  Hitzestufe 6–7/9 zum Anbraten und 4/9 zum köcheln lassen.

  \freeform{}\textbf{Haltbarkeit:}
  Hält sich im Kühlschrank mehrere Wochen.

  \freeform{}\textbf{Tipp:}
  Geschmack kann durch die Verwendung von unterschiedlichen Chilis variiert werden.

\end{recipe}
