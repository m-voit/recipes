\begin{recipe}{Panna Cotta} {5 Portionen} {?? Minuten}

  \freeform
  \textit{Panna Cotta.}

  \ingredient[6]{Blatt}{Gelatine}
  \ingredient[500]{g}{Sahne}
  \ingredient[40]{g}{Zucker}
  \ingredient[1]{}{Vanilleschote}

  Förmchen vorbereiten.
  Die Gelatine in kaltem Wasser einweichen.
  Sahne mit Zucker und der aufgeschlitzten und ausgekratzten Vanilleschote aufkochen und etwa 10 Minuten leise köcheln lassen.

  \newstep
  Die Sahne vom Herd ziehen und die Vanilleschote entfernen.
  Die aufgeweichten Gelatine Blättchen einzeln aus dem Wasser nehmen, gut ausdrücken und unter Rühren im heißen Sahne-Mix auflösen.

  \newstep
  Die Creme in die vorbereiten Förmchen füllen, mit Frischhaltefolie abdecken und für mindestens 4 Stunden in den Kühlschrank stellen.
  Creme fest werden lassen.

  \freeform
  \hrulefill

  \freeform
  \textbf{Tipp:}
  Mit einer fruchtig, sauren Sauce servieren. Varianten auf der nächsten Seite.

  \newpage

  \freeform
  \textit{Lemon-Curd-Sauce passend zu Panna Cotta. 5 Portionen}

  \ingredient[1]{Msp}{Zitronenschale}
  \ingredient[70]{ml}{Zitronensaft}
  \ingredient[1]{}{Ei}
  \ingredient[1]{}{Eigelb}
  \ingredient[58]{g}{Zucker}
  \ingredient[63]{g}{Butter}
  \ingredient[50]{ml}{Maracujasaft}
  \ingredient[]{}{Nüsse und Pistazien, gehackt}

  Eier, Eigelb, Zucker und Butter in einen Topf geben.
  Zitronensaft und Zitronenschale dazugeben.
  Anschließend alles bei mittlerer Hitze unter Rühren erwärmen.
  Die Masse unterrühren und so lange erwärmen bis sie dick cremig ist.
  Dabei die Masse nicht zu heiß werden lassen, da sonst das Ei gerinnt.

  \newstep
  Sauce durch ein nicht zu feines Sieb streichen.
  Den Maracujasaft unterrühren und die Sauce auskühlen lassen.

  \newstep
  Mit Nüssen und Pistazien bestreuen und servieren.

  \freeform
  \hrulefill

  \freeform
  \textit{Beerensauce passend zu Panna Cotta.}

  \ingredient[200]{ml}{Wasser}
  \ingredient[50]{ml}{Zitronensaft}
  \ingredient[80]{g}{Broombeeren}
  \ingredient[130]{g}{Himbeeren}
  \ingredient[30]{g}{Heidelbeeren}
  \ingredient[50]{g}{Zucker}
  \ingredient[1]{}{Zimtstange}
  \ingredient[2]{}{Nelken}

  Zutaten vermischen und aufkochen.

  \freeform
  \hrulefill

\end{recipe}
