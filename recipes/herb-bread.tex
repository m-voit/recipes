\begin{recipe}{Kräuter-Faltenbrot} {8 Portionen} {60 Minuten Arbeitszeit, 165 Minuten Wartezeit, 225 Minuten Gesamt}

  \freeform{}\textit{Beilage zum Grillen.}

  \ingredient[300]{ml}{Wasser lauwarm}
  \ingredient[1]{Würfel}{Hefe}
  \ingredient[2]{TL}{Zucker}
  \ingredient[600]{g}{Mehl}
  \ingredient[2]{TL}{Salz}
  \ingredient[50]{ml}{Öl}

  \textbf{30 min}
  Die Hefe in kleinen Stücken in lauwarmes Wasser geben und im Wasser auflösen.
  Danach den Zucker ins Wasser geben.
  Anschließend Mehl, Salz und Öl zugeben und zu einem Teig verkneten.

  \newstep{}\textbf{120 min}
  Den Teig in eine Schüssel geben und mit einem feuchten Küchentuch abgedeckt für 2h bei 50 °C gehen lassen.

  \ingredient[125]{g}{Kräuterbutter}
  \textbf{30 min}
  
  Dann Teig dann zu einem Viereck ausrollen und mit der Kräuterbutter bestreichen.
  Das Viereck zu gleichmäßigen Streifen schneiden und diese dann in Falten legen.
  Die gefalteten Streifen dann von der Mitte ausgehend in eine sehr gut mit Backpapier ausgelegte Springform stellen (hochkant).

  \newstep{}\textbf{45 min}
  Das Ganze für 15 Minuten gehen lassen.
  Danach bei 180 °C Umluft oder 200° Ober-/Unterhitze für 30--35 Minuten im Ofen backen.

  \freeform{}\hrulefill{}

  \freeform{}\textbf{Tipp:}
  Kräuterbutter aus dem Bereich Butter verwenden.
  Stufe 1--3/4 der Küchenmaschine für das Rühren des Teiges verwenden.
  Teig auf eine Fläche von 30 × 40 cm und eine Dicke von 2 mm ausrollen.
  Teig in 6 cm breite Streifen schneiden.

\end{recipe}
