\begin{recipe}{Semmelknödel} {6 Stück} {30 Minuten Arbeitszeit, 35 Minuten Wartezeit, 105 Minuten Gesamt}

  \freeform{}\textit{Ergibt 6 Semmelknödel.}

  \ingredient[250]{g}{Semmelwürfel}
  \ingredient[250]{ml}{Milch}
  \ingredient[1]{x}{Zwiebel weiß}
  \ingredient[30]{g}{Butter}
  \ingredient[3]{x}{Eier}
  \ingredient[1]{x}{Petersilie}
  \ingredient[1]{Prise}{Salz}
  \ingredient[1]{Prise}{Pfeffer}

  Zuerst Semmelwürfel mit lauwarmer Milch übergießen und leicht durchmischen.
  Die Knödelmasse für 10 Minuten ziehen lassen.
  Danach die gehackte Zwiebel andünsten, die Butter zerlassen und die Eier aufschlagen.
  Zwiebel, Butter und Eier zur Knödelmasse hinzugeben und gut vermengen.
  Die Knödelmasse mit Salz und Pfeffer würzen und nochmal 10 Minuten ziehen lassen.
  Anschließend mit nassen Händen 6 Knödel formen.

  \newstep{}\textbf{15 min}
  Einen Topf mit Wasser bis zum Kochen erhitzen.
  Anschließend Salz hinzugeben und zurückschalten bis das Wasser kurz vorm Kochen ist.
  Knödel ins Wasser geben und 15 Minuten ziehen lassen.
  Zum Schluss Knödel abseihen und warm servieren.

  \freeform{}\hrulefill{}

  \freeform{}\textbf{Hitzestufe:}
  Hitzestufe 9/9 zum Erhitzen und 6/9 zum Ziehen lassen.

\end{recipe}
