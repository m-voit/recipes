\begin{recipe}{BBQ-Sauce mit Kellerbier} {500ml} {60 Minuten Arbeitszeit, 60 Minuten Wartezeit, 120 Minuten Gesamt}

  \freeform{}\textit{Ergibt eine scharf würzige BBQ-Sauce mit malziger Bier Note.}

  \ingredient[1]{x}{Zwiebel rot, mittelgroß}
  \ingredient[2]{x}{Knoblauchzehe}
  \ingredient[4]{EL}{Olivenöl}
  \ingredient[340]{ml}{Kellerbier süffig}

  Zunächst Zwiebel und Knoblauch abziehen und in feine Würfel schneiden.
  Zwiebel und Knoblauch im erhitzten Öl glasig anschwitzen.
  Das Ganze dann mit dem Kellerbier ablöschen.

  \ingredient[400]{ml}{Tomaten passiert}
  \ingredient[8]{EL}{Honig}
  \ingredient[4]{EL}{Worcester Sauce}
  \ingredient[100]{g}{Tomatenmark}
  \ingredient[2]{TL}{Dijon-Senf}

  Anschließend die passierten Tomaten, den Honig, die Worcester Sauce, das Tomatenmark und den Senf hinzufügen.

  \ingredient[1]{TL}{Salz}
  \ingredient[1]{TL}{Pfeffer}
  \ingredient[\fr12]{TL}{Paprikapulver geräuchert}
  \ingredient[\fr12]{TL}{Chipotle Chili}

  Den Pfeffer, die Chipotle Chili, die geräucherte Paprika und das Salz hinzugeben.
  Dann kurz aufkochen lassen.

  \newstep{}\textbf{20 min}
  Bei geringer Hitze die Sauce etwa 20 Minuten köcheln lassen.
  Zum Abschluss die Sauce abschmecken und nach Bedarf nachwürzen.
  Anschließend über Nacht ziehen lassen.

  \freeform{}\hrulefill{}

  \freeform{}\textbf{Hitzestufe:}
  Hitzestufe 6–7/9 zum Anbraten und 4/9 zum köcheln lassen.

  \freeform{}\textbf{Haltbarkeit:}
  Hält sich im Kühlschrank mehrere Wochen.

  \freeform{}\textbf{Tipp:}
  Zur Verwendung als Glasur für Grillgut die Sauce weniger lange einkochen lassen, sodass sie eine dünnflüssige Konsistenz behält.

\end{recipe}
