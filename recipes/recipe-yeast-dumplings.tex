\begin{recipe}{Germknödel} {6 Stück} {30 Minuten Arbeitszeit, 135 Minuten Wartezeit, 165 Minuten Gesamt}

  \freeform{}\textit{Ergibt sechs Germknödel. Mit Vanillesauce servieren.}

  \ingredient[550]{g}{Mehl}
  \ingredient[1]{Prise}{Salz}
  \ingredient[30]{g}{Hefe}
  \ingredient[60]{g}{Zucker}
  \ingredient[1]{Pck.}{Vanillezucker}
  \ingredient[70]{g}{Butter}
  \ingredient[300]{ml}{Milch}
  \ingredient[1]{x}{Eigelb}
  \ingredient[1]{x}{Ei}
  \ingredient[300]{g}{Pflaumenmarmelade}
  \ingredient[2-3]{EL}{Rum, dunkel}
  \ingredient[\fr12]{}{Schale einer Zitrone, fein gerieben}

  Mehl in eine große Schüssel sieben und Salz dazugeben.

  \newstep{}\textbf{Dampferl ansetzen}
  Dabei 50ml der Milch mit einem Esslöffel Zucker auf ca. 45 Grad erwärmen und darin die Hefe auflösen.
  Zu beachten ist, dass über 45 Grad die Hefe kaputt geht.
  Eine Mulde in das Mehl drücken, das Dampferl hineinschütten und 15min an einem warmen Ort gehen lassen bis die Hefe stark aufschäumt.

  \newstep{}Restliche Zutaten: Die Butter in einem Topf bei niedriger Hitze zergehen lassen.
  Anschließend Milch, Zucker, Vanillezucker, Zitronenschale und Eier dazugeben.
  Die Mischung leicht anwärmen und mit einem Schneebesen leicht schaumig schlagen.

  \newstep{}\textbf{Teig:}
  Diese Mischung zum Mehl geben, wenn das Dampferl fertig ist und alles gut verkneten, bis der Teig schön gleichmäßig und glatt ist.
  Den Teig in sechs gleich große Stücke schneiden und mit etwas Mehl auf einer großen mit Mehl bestäubten Platte schleifen.
  Anschließend mit einem Küchentuch zudecken und 60min an einem warmen Ort gehen lassen.

  \newstep{}\textbf{Füllung:}
  Marmelade mit Rum vermengen.
  Eine Mulde in die Knödel drücken und jeweils ein Sechstel der Marmelade hineingeben.
  Knödel nach oben ziehen und zusammenklappen und mit je zwei Fingern festdrücken und gut verschließen.
  Mit der Naht nach unten wieder auf die mit Mehl bestäubte Platte legen, erneut zudecken und 60min gehen lassen.

  \newstep{}\textbf{Kochen:}
  Einen Topf mit Dampfgareinsatz soweit mit Wasser füllen, dass das siedende Wasser nicht den Siebeinsatz berührt.
  Mit einem Deckel zudecken und warten bis das Wasser kocht.
  Das Sieb vor jedem Knödel mit Öl einfetten!
  Die Knödel so hineinsetzen, dass sie sich nicht berühren und sie Platz haben nochmal aufzugehen.
  Die Garzeit beträgt 12min.

  \freeform{}\hrulefill{}

  \freeform{}\textbf{Tipp:}
  Mit Vanillesauce servieren und mit Mohn bestäuben.

  \end{recipe}
