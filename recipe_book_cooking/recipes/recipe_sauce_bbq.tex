\begin{recipe}{BBQ-Sauce} {750ml} {60 Minuten}

\freeform
\textit{Ergibt eine würzige und leicht scharfe BBQ-Sauce mit Whiskey-Aroma.}

\ingredient[1]{x}{Zwiebel}
\ingredient[2]{x}{Knoblauchzehe}
\ingredient[1]{x}{Apfel}
\ingredient[1]{EL}{Olivenöl}

Zuerst Zwiebel, Knoblauchzehen und Apfel schälen und fein würfeln.
Olivenöl in einer beschichteten Pfanne erhitzen und die Zwiebel darin anbraten.

\ingredient[400]{g}{Tomaten stückig}
\ingredient[100]{g}{Tomatenmark}
\ingredient[125]{g}{Rohrzucker}
\ingredient[150]{ml}{Apfelessig}
\ingredient[50]{ml}{Worcestersauce}
\ingredient[50]{ml}{Whiskey rauchig}

Nach ca. 1 Minute Knoblauch, Apfel und Tomaten zufügen.
Nach weiteren 2 Minuten Tomatenmark, Rohrzucker, Apfelessig, Worcestersauce und Whiskey zufügen und gut verrühren.

\ingredient[\fr12]{TL}{Chilipulver}
\ingredient[2]{TL}{Paprikapulver rosenscharf}
\ingredient[\fr12]{TL}{Pfeffer schwarz}
\ingredient[1]{TL}{Meersalz}

Dann 1/2 TL Chilipulver, 2 TL Paprikapulver, 1/2 TL Pfeffer und 1 TL Meersalz zugeben.

\newstep
Die Sauce ca. 30 Min. bei schwacher Hitze köcheln lassen.
Sauce pürieren und sofort in eine saubere Flasche füllen.
Anschließend über Nacht ziehen lassen.

\freeform
\hrulefill

\freeform
\textbf{Hitzestufe:}
Hitzestufe 6-7 zum Anbraten und Hitzestufe 4 zum köcheln lassen.

\freeform 
\textbf{Haltbarkeit:}
Hält sich im Kühlschrank mehrere Wochen.

\freeform 
\textbf{Tipp:}
Geschmack kann durch die Verwendung von unterschiedlichen Whiskys variiert werden.

\freeform 
\textbf{Modifikationen:}
Rohrzucker von 200g auf 125g reduziert. Rapsöl durch Olivenöl ersetzt. Tomatenmenge von 300g auf 400g erhöht.

\end{recipe}
