\documentclass[
  DIV=11,% DIV Faktor für Satzspiegelberechnung - muss bei anderen Schriftgrößen als 11pt angepasst werden , sie Doku zu KOMA Script
  pagesize,% write pagesize to DVI or PDF
  fontsize=11pt,% use this font size
  paper=a4,% use ISO A4
]{scrartcl}

\usepackage[utf8]{inputenc}
\usepackage[sfdefault, scaled=.95, book]{FiraSans}
\usepackage[T1]{fontenc}
\usepackage{textcomp}
\usepackage{graphicx}
\usepackage{caption}
\usepackage{subcaption}
\usepackage[ngerman]{babel}
\usepackage{microtype}% optischer Randausgleich etc.
\usepackage[nonumber]{cuisine}
\usepackage{nicefrac}

\selectlanguage{ngerman}
\pagenumbering{gobble}

\begin{document}
\begin{recipe}{BBQ-Butter} {125g} {10 Minuten}

\freeform
\textit{Ergibt eine würzig scharfe BBQ-Butter.}

\ingredient[\fr12]{Stück}{Butter (125g)}
\ingredient[50]{g}{Paprikamark oder Ajvar}
\ingredient[\fr12]{EL}{Worcestersauce}
\ingredient[\fr12]{EL}{Senf scharf}
\ingredient[1\fr12]{EL}{Paprikapulver}
\ingredient[\fr12]{TL}{Salz}
\ingredient[\fr12]{TL}{Paprikaflocken}

Zuerst die weiche Butter cremig schlagen.
Alle weiteren Zutaten zugeben und gut vermischen, bis eine homogene Masse entsteht.

\ingredient[]{}{Pfeffer}
\ingredient[]{}{Cayennepfeffer}
\ingredient[]{}{Chilipulver}

Anschließend mit Pfeffer und Chilipulver, optional mit Cayennepfeffer, abschmecken.

\newstep
Über Nacht im Kühlschrank ziehen lassen.

\freeform
\hrulefill

\freeform 
\textbf{Haltbarkeit:}
Hält sich im Kühlschrank ca. 1 Woche.

\freeform 
\textbf{Tipp:}
Paprikapulver für die Butter aus Paprika edelsüß, Paprika rosenscharf und Pimentón de la Vera mischen.

\end{recipe}
\end{document}
