\documentclass[
  DIV=11,% DIV Faktor für Satzspiegelberechnung - muss bei anderen Schriftgrößen als 11pt angepasst werden , sie Doku zu KOMA Script
  pagesize,% write pagesize to DVI or PDF
  fontsize=11pt,% use this font size
  paper=a4,% use ISO A4
]{scrartcl}

\usepackage[utf8]{inputenc}
\usepackage[sfdefault, scaled=.95, book]{FiraSans}
\usepackage[T1]{fontenc}
\usepackage{textcomp}
\usepackage{graphicx}
\usepackage{caption}
\usepackage{subcaption}
\usepackage[ngerman]{babel}
\usepackage{microtype}% optischer Randausgleich etc.
\usepackage[nonumber]{cuisine}
\usepackage{nicefrac}

\selectlanguage{ngerman}
\pagenumbering{gobble}

\begin{document}
\begin{recipe}{Chipotle Chili Butter} {125g} {10 Minuten}

\freeform
\textit{Ergibt eine würzig scharfe Chipotle Chili Butter.}

\ingredient[\fr12]{Stück}{Butter (125g)}
\ingredient[\fr12]{TL}{Chipotle Chili}
\ingredient[\fr14]{TL}{Paprikapulver geräuchert}
\ingredient[\fr14]{TL}{Chilipulver}
\ingredient[1]{Prise}{Salz}

Zuerst Chipotle Chili Flocken mit einem Mörser zerkleinern.
Anschließend die restlichen Zutaten mit der zimmerwarmen Butter vermengen.
Nach Bedarf mit Salz abschmecken.

\newstep
Im Kühlschrank über Nacht ziehen lassen

\freeform
\hrulefill

\freeform 
\textbf{Haltbarkeit:}
Hält sich im Kühlschrank ca. 1 Woche.

\freeform 
\textbf{Modifikationen:}
Einen Teelöffel Chiliflocken hinzufügen.

\end{recipe}
\end{document}
