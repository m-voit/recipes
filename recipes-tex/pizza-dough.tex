\documentclass[../recipe-collections/cooking.tex]{subfiles}

\begin{document}
\begin{recipe}{Pizzateig}{6–8 Stück \faIcon{utensils}}{30h \faIcon{user-clock}, 30min \faIcon{user-cog}}

  \freeform{}\textit{Teig für 6–8 Pizzen.}

  \ing[570]{g}{Wasser}
  \ing[12,75]{g}{Frischhefe}
  \ing[1]{kg}{Caputo Pizzamehl}

  Zunächst die Hefe zerbröseln und in lauwarmen, maximal 37,5 °C, Wasser auflösen.
  Anschließend das Mehl rühren und das Wasser langsam unter Rühren hinzugeben.
  Dabei zuerst 570g Wasser hinzugeben und nur bei Bedarf die restlichen 10g Wasser hinzugeben.

  \ing[30]{ml}{Olivenöl}
  \ing[34,5]{g}{Salz}

  Den Teig mindestens 5 Minuten lang glatt kneten, dann langsam das Öl unterkneten.
  Sobald das Öl untergeknetet ist, am Ende das Salz unterkneten.

  \newstep{}\textbf{24-48h}
  Den Teig luftdicht in einer Schüssel, die doppelt so groß ist wie der Teig, im Kühlschrank ruhen lassen.
  Der Teig muss dabei immer abgedeckt sein.

  \newstep{}Nach der Gehzeit im Kühlschrank den Teig in 220-280g große Teilstücke teilen.
  Die Größe ist abhängig von der Größe des Pizzasteins.
  Den Teig dabei zu Kugeln formen, den Teig schleifen und richtig durchkneten.

  \newstep{}\textbf{6h}
  Nun den Teig 6h luftdicht bei Raumtemperatur ruhen lassen.

  \freeform{}\hrulefill{}

  \freeform{}\faIcon{lightbulb}
  Einen Teelöffel Honig hinzufügen, damit der Teig auch im Elektrobackofen knusprig wird.

\end{recipe}
\end{document}
