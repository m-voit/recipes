\documentclass[../recipe-collections/cooking.tex]{subfiles}

\begin{document}
\begin{recipe}{Süßkartoffel-Erdnuss-Eintopf}{4 Portionen \faIcon{utensils}}{60min \faIcon{user-clock}, 25min \faIcon{user-cog}}

  \freeform{}\textit{}

  \ing[1]{x}{Zwiebel, groß}
  \ing[1]{x}{Stange Sellerie}
  \ing[400]{g}{Süßkartoffeln}
  \ing[150]{g}{Kartoffeln}
  \ing[2]{x}{Knoblauchzehen}

  \textbf{25min}
  Zwiebel schälen, halbieren und in feine Streifen schneiden.
  Sellerie putzen, waschen und in Stücke schneiden.
  Süßkartoffeln und Kartoffeln schälen und würfeln.
  Knoblauch schälen und fein hacken.

  \ing[2]{EL}{Öl}
  \ing[2]{x}{Lorbeerblätter}

  \textbf{10min}
  Öl in einem großen Topf erhitzen, Zwiebeln darin bei schwacher Hitze glasig anbraten.
  Lorbeerblätter zugeben und mitschmoren.
  Sellerie, beide Kartoffelsorten und Knoblauch zugeben und alles ca. 5 Minuten anbraten.

  \ing[150]{g}{Berglinsen}
  \ing[800]{g}{Tomaten, gehackt}
  \ing[850]{ml}{Wasser}
  \ing[2]{TL}{Kreuzkümmel}
  \ing[]{}{Salz}
  \ing[]{}{Pfeffer}
  \ing[2]{EL}{Erdnussbutter}
  \ing[300]{g}{Kichererbsen, abgetropft}

  \textbf{50min}
  Linsen zugeben, kurz mitbraten, dann mit Tomaten und 850 ml Wasser ablöschen.
  Mit Salz, Pfeffer und Kreuzkümmel abschmecken, Erdnussbutter unterrühren und alles ca. 50 Minuten köcheln lassen.
  Kurz vor Ende der Garzeit die abgegossenen Kichererbsen unterrühren.

  \ing[1]{x}{Limette}

  Limettensaft auspressen und Eintopf mit Limettensaft abschmecken.

  \ing[\fr12]{Bund}{Petersilie}
  \ing[100]{g}{Joghurt}

  Petersilie waschen, trocken schütteln und hacken.
  Eintopf mit Petersilie und Joghurt anrichten.

  \freeform{}\hrulefill{}

  \freeform{}\faIcon{burn}
  Stufe 5/9 zum Anbraten und Stufe 6/9 zum köcheln lassen.

  \freeform{}\faIcon{info}
  Nährwerte pro Portion: 456kcal, 73,2g Kohlenhydrate, 13,7g Fett, 16,1g Eiweiß, 0,38g Salz

\end{recipe}
\end{document}
