\documentclass[../recipe-collections/cooking.tex]{subfiles}

\begin{document}
\begin{recipe}{Kräuterbutter}{250g \faIcon{utensils}}{12h \faIcon{user-clock}, 10min \faIcon{user-cog}}

  \freeform{}\textit{Ergibt eine klassische Kräuterbutter.}

  \ing[1]{Stück}{Butter (250g)}
  \ing[1]{Bund}{Petersilie}
  \ing[1]{Bund}{Schnittlauch}
  \ing[2]{x}{Knoblauchzehe}
  \ing[1]{Prise}{Salz}

  \textbf{10min}
  Zuerst Petersilie, Schnittlauch und Knoblauchzehe klein hacken.
  Anschließend alle Zutaten mit der zimmerwarmen Butter vermengen.
  Nach Bedarf mit Salz abschmecken.

  \newstep{}\textbf{12h}
  Im Kühlschrank über Nacht ziehen lassen

  \freeform{}\hrulefill{}

  \freeform{}\faIcon{lightbulb}
  Ein Bund entspricht ca. 30g gefriergetrockneter Kräuter.

  \freeform{}\faIcon{snowflake}
  Hält sich im Kühlschrank ca. 4 Wochen.

\end{recipe}
\end{document}
