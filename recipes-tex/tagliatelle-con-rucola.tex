\documentclass[../recipe-collections/cooking.tex]{subfiles}

\begin{document}
\begin{recipe}{Tagliatelle con Rucola}{4 Portionen \faIcon{utensils}}{20min \faIcon{user-cog}}

  \freeform{}\textit{Sommerliche Nudelpfanne mit Rucola und aromatischen Dörrtomaten.}

  \ing[400]{g}{Tagliatelle}
  \ing[100]{ml}{Nudelwasser}

  Tagliatelle nach Packungsanleitung al dente kochen.
  Beim Abgießen 100 ml Nudelwasser auffangen.

  \ing[60]{g}{Pinienkerne}

  Die Pinienkerne in einer beschichteten Pfanne ohne Fett bei mittlerer Hitze goldgelb anrösten.
  Dabei gelegentlich umrühren, damit die Kerne nicht verbrennen.
  Anschließend herausnehmen und beiseite stellen.

  \ing[2]{x}{Handvoll Rucola}
  \ing[200]{g}{getrocknete Tomaten in Öl}
  \ing[2]{x}{Knoblauchzehen}
  \ing[1]{EL}{Olivenöl}

  \textbf{3min}
  Den Rucola waschen und trocken schleudern.
  Die getrockneten Tomaten aus dem Öl nehmen und in feine Streifen schneiden.
  Den Knoblauch schälen und fein hacken.
  Das Olivenöl in einer Pfanne erhitzen und den Knoblauch sowie die Tomaten 3 Minuten bei mittlerer Hitze anbraten.

  \ing[1]{EL}{Kapern}
  \ing[1]{Pr.}{Salz}
  \ing[4]{TL}{Grünes Pestogewürz}

  \textbf{1min}
  Die Kapern hinzufügen und mit dem grünen Pestogewürz und einer Prise Salz würzen.
  Die gekochten Tagliatelle und das Nudelwasser hinzufügen.
  Anschließend 1 Minute köcheln lassen.
  Rucola und Pinienkerne unterheben und sofort servieren.

  \freeform{}\hrulefill{}

  \freeform{}\faIcon{lightbulb}
  Als kalten Nudelsalat genießen. Dazu abkühlen lassen und 1 EL Olivenöl sowie 1 EL Balsamicoessig hinzufügen.

  \freeform{}\faIcon{burn}
  Stufe 5/9 zum Rösten und Stufe 6/9 zum Anbraten.

  \freeform{}\faIcon{info}
  Nährwerte pro Portion: 344kcal, 34g Kohlenhydrate, 18g Fett, 8,2g Eiweiß, 2,2g Salz

\end{recipe}
\end{document}
