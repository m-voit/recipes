\documentclass[../recipe-collections/cooking.tex]{subfiles}

\begin{document}
\begin{recipe}{Nudelteig}{4 Portionen \faIcon{utensils}}{4h \faIcon{user-clock}, 30min \faIcon{user-cog}}

  \freeform{}\textit{}

  \ing[250]{g}{Mehl Type 405}
  \ing[250]{g}{Semola}
  \ing[1]{x}{Ei}
  \ing[1]{EL}{Olivenöl}
  \ing[1]{Prise}{Salz}

  Alle Zutaten bis auf das Wasser krümelig verrühren.

  \ing[]{}{Wasser}

  Danach mit einem Teelöffel unter ständigem Rühren das Wasser dazugeben.
  Soviel Wasser verwenden, bis die Grießkörner nicht mehr zu sehen sind.
  Der Teig darf nicht klebrig sein und bleibt in der Küchenmaschine krümelig.

  \newstep{}\textbf{4h}
  Anschließend den Teig gut mit der Hand durchkneten und die entstandene Kugel in Frischhaltefolie 3-4h in den Kühlschrank legen.

  \newstep{}Den Teig so lange durch eine Nudelmaschine geben, bis der Teig fein, glatt und elastisch ist, dann erst die Nudeln machen.

  \freeform{}\hrulefill{}

  \freeform{}\faIcon{lightbulb}
  Semola Hartweizengrieß von De Cecco verwenden.

\end{recipe}
\end{document}
