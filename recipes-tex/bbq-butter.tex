\documentclass[../recipe-collections/cooking.tex]{subfiles}

\begin{document}
\begin{recipe}{BBQ-Butter}{250g \faIcon{utensils}}{12h \faIcon{user-clock}, 10min \faIcon{user-cog}}

  \freeform{}\textit{Ergibt eine würzig scharfe BBQ-Butter.}

  \ing[1]{Stück}{Butter (250g)}
  \ing[100]{g}{Paprikamark oder Ajvar}
  \ing[1]{EL}{Worcestersauce}
  \ing[1]{EL}{Senf scharf}
  \ing[3]{EL}{Paprikapulver}
  \ing[1]{TL}{Salz}
  \ing[1]{TL}{Paprikaflocken}

  \textbf{5min}
  Zuerst die weiche Butter cremig schlagen.
  Alle weiteren Zutaten zugeben und gut vermischen, bis eine homogene Masse entsteht.

  \ing[]{}{Pfeffer}
  \ing[]{}{Cayennepfeffer}
  \ing[]{}{Chilipulver}

  \textbf{5min}
  Anschließend mit Pfeffer und Chilipulver, optional mit Cayennepfeffer, abschmecken.

  \newstep{}\textbf{12h}
  Über Nacht im Kühlschrank ziehen lassen.

  \freeform{}\hrulefill{}

  \freeform{}\faIcon{lightbulb}
  Paprikapulver für die Butter aus Paprika edelsüß, Paprika rosenscharf und Pimentón de la Vera mischen.

  \freeform{}\faIcon{snowflake}
  Hält sich im Kühlschrank ca. 2 Wochen.

\end{recipe}
\end{document}
