\documentclass[../recipe-collections/cooking.tex]{subfiles}

\begin{document}
\begin{recipe}{Balsamico-Zwiebeln}{4 Portionen \faIcon{utensils}}{20min \faIcon{user-clock}, 15min \faIcon{user-cog}}

  \freeform{}\textit{Warm oder kalt als Beilage zu gegrilltem, Burgern, Sandwiches oder zu einer Antipasti-Platte.}

  \ing[400]{g}{Zwiebel, rot, klein}
  \ing[2]{EL}{Olivenöl}
  \ing[1]{Msp.}{Salz}
  \ing[1]{Msp.}{Pfeffer}
  \ing[1]{TL}{Thymian}

  \textbf{15min}
  Die Zwiebeln schälen und längs vierteln.
  Das Öl in einer Pfanne erhitzen und die Zwiebelviertel darin etwa 3 Minuten anbraten.
  Mit Salz, Pfeffer und Thymian würzen.

  \ing[2]{EL}{Rohrohrzucker}
  \ing[60]{ml}{Aceto Balsamico}

  \textbf{20min}
  Den Zucker über die Zwiebeln streuen und karamellisieren lassen, dabei ab und zu umrühren.
  Mit Balsamico ablöschen und die Zwiebeln 10 Minuten leicht köcheln lassen.
  Dann vom Herd nehmen, abdecken und noch mindestens 10 Minuten ziehen lassen.

  \freeform{}\hrulefill{}

  \freeform{}\faIcon{lightbulb}
  Die Zwiebeln können gut am Vortag zubereitet werden – ordentlich durchgezogen schmecken sie besonders lecker.

  \freeform{}\faIcon{snowflake}
  Im Kühlschrank halten sich die Balsamico-Zwiebeln etwa 4 Tage.

  \freeform{}\faIcon{info}
  Nährwerte pro Portion: 109kcal, 14g Kohlenhydrate, 5g Fett, 1g Eiweiß, 1g Ballaststoffe

\end{recipe}
\end{document}
