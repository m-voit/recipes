\documentclass[../recipe-collections/cooking.tex]{subfiles}

\begin{document}
\begin{recipe}{Burger Buns} {4 Stück \faIcon{utensils}} {2h 10min \faIcon{user-clock}, 20min \faIcon{user-cog}}

  \freeform{}\textit{Fluffige, rustikale Burgerbrötchen.}

  \ing[240]{g}{Weizenmehl Typ 550}
  \ing[16]{g}{Hefe frisch}
  \ing[20]{g}{Zucker}
  \ing[1]{TL}{Salz}
  \ing[90]{ml}{Milch}
  \ing[2]{x}{Eier}
  \ing[12 \fr12]{g}{Butter}

  \textbf{1h 25min}
  Hefe und Zucker vermengen, damit sich die Hefe sich auflöst.
  Die restlichen Zutaten (ohne Eiweiß) in eine Rührschüssel geben.
  Sobald die Hefe flüssig ist, diese hinzugeben.
  Jetzt den Teig für ca. 15 Minuten durchkneten und anschließend an einem warmen Ort für 60 Minuten gehen lassen.
  Dabei verdoppelt sich das Volumen in etwa.

  \newstep{}\textbf{1h 5min}
  Nun den Teig aus der Schüssel nehmen und in 90g Portionen portionieren.
  Die einzelnen Teiglinge müssen nun geschliffen werden.
  Die geschliffenen Teiglinge auf ein bemehltes Backblech geben und weitere 45 Minuten im 40 Grad warmen Backofen gehen lassen.
  In der Zwischenzeit das übrig gebliebene Eiweiß und Wasser kurz verquirlen.

  \newstep{}\textbf{10 min}
  Wenn die Teiglinge gut aufgegangen sind, diese aus dem Ofen holen und den Ofen auf 205 Grad Ober-/Unterhitze vorheizen.
  Wenn die Temperatur erreicht ist, die Brötchen mit der Eiweißmischung einpinseln und dann für 10 Minuten im Ofen backen.
  Die Teiglinge ab Minute 8 beobachten, damit diese nicht zu dunkel werden.
  Danach abkühlen lassen.

  \freeform{}\hrulefill{}

  \freeform{}\faIcon{lightbulb}
  Gut bemehltes Arbeitsbrett verwenden.
  Eigelb in den Teig, Eiweiß zum Einpinseln.
  Backpapier verwenden.

\end{recipe}
\end{document}
