\documentclass[../recipe-collections/cooking.tex]{subfiles}

\begin{document}
\begin{recipe}{Bayerische Dampfnudeln}{6 Portionen \faIcon{utensils}}{2h 10min \faIcon{user-clock}, 30min \faIcon{user-cog}}

  \freeform{}\textit{Klassische bayerische Dampfnudeln.}

  \ing[500]{g}{Mehl}
  \ing[1]{x}{Würfel Hefe}
  \ing[\fr14]{l}{Milch}
  \ing[1]{EL}{Zucker}
  \ing[1]{TL}{Salz}
  \ing[1]{x}{Ei}

  \textbf{15min}
  Mehl in Schüssel sieben.
  Die Hefe zerbröseln, mit dem Zucker, etwas lauwarmer Milch und etwas Mehl zu dickflüssigem Teig verrühren.
  In die Mehlmitte eine Grube machen, die verrührte Hefe hineingeben und leicht mit Mehl überstauben.
  Nun das Hefeteiglein in zugedeckter Schüssel etwa 10 bis 15min lauwarm gehen lassen, bis es die doppelte Größe erreicht hat.

  \newstep{}\textbf{1h}
  Gegangenen Vorteig mit etwas Mehl verrühren, lauwarme Milch, Eier, Zucker und Salz unterrühren.
  Danach den Teig so lange abschlagen, dabei eventuell mit einem Rührgerät kneten, bis er sich von der Schüssel löst, eine glatte und gleichmäßige Beschaffenheit hat und Blasen wirft.
  Nun den Teig wieder zugedeckt etwa 1 Stunde gehen lassen.

  \newstep{}\textbf{30min}
  Gegangenen Teig zu Nudeln formen, etwa 8-11 Stück je nach Größe und nochmal auf bemehltem Brett 30 Minuten ruhen lassen.

  \ing[\fr14]{l}{Milch}
  \ing[50]{g}{Butter}
  \ing[1]{EL}{Zucker}

  \textbf{25min}
  In der Zwischenzeit die Milch, die Butter und Zucker in einer Bräterpfanne aufkochen lassen.
  Anschließend die Nudeln in die kochende Milch geben und sofort auf die kleinste Wärmestufe zurückschalten.
  Bei schwacher Hitze, die Nudeln müssen leise köcheln, 20-25min köcheln lassen, bis alle Milch aufgesogen und die Krustenbildung mit \textit{Singen} oder \textit{Krachen} beginnt.

  \newstep{}Nun die Nudeln aus der Pfanne nehmen und mit Kompott, Vanillesauce oder Weinschaumsauce servieren.

  \freeform{}\hrulefill{}

  \freeform{}\faIcon{lightbulb}
  Eine Pfanne mit 28cm Durchmesser und 7cm Tiefe verwenden. Während des Garens darf der Deckel nicht abgenommen werden!

\end{recipe}
\end{document}
