\documentclass[../recipe-collections/cooking.tex]{subfiles}

\begin{document}
\begin{recipe}{Kräuter-Faltenbrot}{8 Portionen \faIcon{utensils}}{2h 45min \faIcon{user-clock}, 60min \faIcon{user-cog}}

  \freeform{}\textit{Beilage zum Grillen.}

  \ing[300]{ml}{Wasser lauwarm}
  \ing[1]{Würfel}{Hefe}
  \ing[2]{TL}{Zucker}
  \ing[600]{g}{Mehl}
  \ing[2]{TL}{Salz}
  \ing[50]{ml}{Öl}

  \textbf{30min}
  Die Hefe in kleinen Stücken in lauwarmes Wasser geben und im Wasser auflösen.
  Danach den Zucker ins Wasser geben.
  Anschließend Mehl zugeben und zu einem Teig verkneten.
  Zuletzt Salz und Öl zugeben und einkneten.

  \newstep{}\textbf{2h}
  Den Teig in eine Schüssel geben und mit einem feuchten Küchentuch abgedeckt für 2h bei 50 °C gehen lassen.

  \ing[125]{g}{Kräuterbutter}

  \textbf{30min}
  Dann Teig dann zu einem Viereck mit 30 x 40cm und einer Dicke von 2mm ausrollen und mit der Kräuterbutter bestreichen.
  Das Viereck zu gleichmäßigen, ca. 6cm breiten Streifen schneiden und diese dann in Falten legen.
  Die gefalteten Streifen dann von der Mitte ausgehend in eine sehr gut mit Backpapier ausgelegte Springform stellen (hochkant).

  \newstep{}\textbf{45min}
  Das Ganze für 15 Minuten gehen lassen.
  Danach bei 180 °C Umluft oder 200° Ober-/Unterhitze für 30--35 Minuten im Ofen backen.

  \freeform{}\hrulefill{}

  \freeform{}\faIcon{lightbulb}
  Stufe 1–3/4 der Küchenmaschine für das Rühren des Teiges verwenden.

\end{recipe}
\end{document}
