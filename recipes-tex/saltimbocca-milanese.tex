\documentclass[../recipe-collections/cooking.tex]{subfiles}

\begin{document}
\begin{recipe}{Saltimbocca alla Milanese}{4 Portionen \faIcon{utensils}}{5min \faIcon{user-clock}, 25min \faIcon{user-cog}}

  \freeform{}\textit{Das originale Saltimbocca-Rezept der römischen Küche.}

  \ing[16]{Blätter}{Salbei}

  Die Salbeiblätter waschen, trocken tupfen und beiseite legen.

  \ing[8]{Scheiben}{Parmaschinken}
  \ing[8]{x}{Kalbsschnitzel, 80g}
  \ing[8]{Pr.}{Gewürz}
  \ing[]{}{Salz}
  \ing[]{}{Pfeffer}

  Den Parmaschinken auf einem Brett auslegen und mit jeweils 1 Blatt Salbei belegen.
  Je ein Kalbsschnitzel darauflegen und mit jeweils 1 Prise Gewürz, Salz und Pfeffer würzen.
  Jedes Schnitzel mit einem weiteren Salbeiblatt belegen.
  Die Schnitzel aufrollen, sodass sie mit dem Parmaschinken komplett umwickelt sind und jeweils mit einem Zahnstocher oder Spieß fixieren.

  \ing[1]{EL}{Olivenöl}
  \ing[50]{ml}{Weißwein}
  \ing[1]{TL}{Balsamicoessig, dunkel}
  \ing[100]{ml}{Wasser}

  Das Olivenöl in einer Pfanne erhitzen und die Kalbsschnitzel bei starker Hitze 2 Minuten von jeder Seite anbraten.
  Mit Weißwein und Balsamicoessig ablöschen.
  Mit Wasser auffüllen und 2 Minuten bei leichter Hitze köcheln lassen.
  Die Kalbsschnitzel herausnehmen und auf einem Teller anrichten.

  \ing[20]{g}{Butter, kalt}

  Für die Sauce die Flüssigkeit in der Pfanne bei starker Hitze um die Hälfte einkochen lassen, vom Herd nehmen und die kalte Butter einrühren.
  Den entstandene Bratenfond mit 1 Prise Salz würzen, über das Fleisch gießen und sofort servieren.

  \freeform{}\hrulefill{}

  \freeform{}\faIcon{lightbulb}
  Anstatt Kalbfleisch kann auch Hähnchen verwendet werden.

\end{recipe}
\end{document}
