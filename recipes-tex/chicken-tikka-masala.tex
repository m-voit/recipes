\documentclass[../recipe-collections/cooking.tex]{subfiles}

\begin{document}
\begin{recipe}{Chicken Tikka Masala} {4 Portionen \faIcon{utensils}} {6h \faIcon{user-clock}, 20min \faIcon{user-cog}}

  \freeform{}\textit{Hühnchen in würziger Tomatensauce nach indischer Art.}

  \ing[400]{g}{Hähnchenbrust}
  \ing[250]{ml}{Naturjoghurt}

  \textbf{6h}
  Die Hühnerbrust grob würfeln und in einer Schüssel mit dem Joghurt vermengen und abgedeckt für 4-6 Stunden oder über Nacht im Kühlschrank einlegen.

  \ing[30]{g}{Ingwer}
  \ing[1]{x}{Zwiebel}
  \ing[200]{g}{Fleischtomate}
  \ing[2]{x}{Chilischote}
  \ing[2]{x}{Knoblauchzehe}

  Den Ingwer und die Zwiebeln schälen und die Tomaten, Zwiebeln und den Ingwer in feine Würfel schneiden.
  Die Kerne der Chilischoten entfernen und klein hacken und den Knoblauch in Würfel schneiden.

  \ing[2]{TL}{Koriandersamen}
  \ing[2]{TL}{Senfkörner}

  Die Koriandersamen, Senfkörner, Chilischote und die Knoblauchzehen in einer Pfanne anrösten.

  \ing[]{}{Salz}

  Die Gewürzmischung, Tomaten, Zwiebeln, Knoblauch, Ingwer und Chili mit etwas Salz in einen Mörser geben und zu einer Paste mörsern.

  \ing[2]{EL}{Pflanzenöl}
  \ing[1]{EL}{Tomatenmark}

  Die Hühnerbrust aus dem Joghurt nehmen, trocken tupfen und in einem flachen Topf in etwas Pflanzöl anbraten. Die Gewürzpaste und Tomatenmark dazugeben und mit einkochen lassen.

  \ing[200]{g}{passierte Tomate}

  \textbf{10min}
  Mit den passierten Tomaten ablöschen und für ca. 10 Minuten bei mittlerer Hitze zu einer Sauce einkochen.

  \ing[240]{g}{Basmati-Reis}

  Währenddessen den Reis in einem Sieb gründlich abwaschen und mit 360 ml Wasser und einem Teelöffel Salz aufsetzten und gar kochen.

  \ing[1]{Bund}{Koriander}
  \ing[5]{EL}{Limettensaft}
  \ing[1]{x}{Limettenabrieb}

  Den Koriander grob hacken und kurz vor dem Anrichten zusammen mit dem Limettensaft und -abrieb unter das Tikka Masala mischen.

  \newstep{}Den Reis auf dem Teller anrichten und mit dem Chicken Tikka Masala und mehr Koriander servieren.

  \freeform{}\hrulefill{}

  \freeform{}\faIcon{lightbulb}
  Alternativ kann zum Chicken Tikka Masala auch indisches Naan-Brot oder Paratha gereicht werden.

  \freeform{}\faIcon{info}
  Nährwerte pro Portion: 448kcal, 59g Kohlenhydrate, 9g Fett, 34g Eiweiß

\end{recipe}
\end{document}
