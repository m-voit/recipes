\documentclass[../recipe-collections/cooking.tex]{subfiles}

\begin{document}
\begin{recipe}{Bifteki Feta} {6 Stück \faIcon{utensils}} {10min \faIcon{user-clock}, 20min \faIcon{user-cog}}
  \freeform{}\textit{Griechisches Bifteki mit Feta Füllung.}

  \ing[\fr12]{x}{altbackenes Brötchen}
  \ing[2]{x}{Knoblauchzehen}
  \ing[1]{x}{Zwiebel}
  \ing[8]{x}{getrocknete Tomaten}
  \ing[10]{g}{Petersilie}
  \ing[100]{g}{Feta}

  \textbf{15min}
  Brötchen in grobe Stücke teilen, in lauwarmes Wasser legen und nach ca. 10 Min. gründlich ausdrücken.
  Knoblauch und Zwiebel schälen, mit getrockneten Tomaten und Petersilie klein hacken und Feta quer in 6 längliche Stücke schneiden.

  \ing[400]{g}{Rinderhackfleisch}
  \ing[1]{x}{Ei}
  \ing[1]{TL}{Kreuzkümmel}
  \ing[1]{TL}{Oregano}
  \ing[1]{Prise}{Salz}
  \ing[1]{Prise}{Pfeffer}

  \textbf{15min}
  Alle Zutaten, bis auf Feta und Rapsöl, gründlich vermischen und zu 6 Bifteki formen.
  Aus dem Hackfleisch einen Kreis (ca. 11 cm Durchmesser, 0,5 cm Dicke) formen und je ein Stücke Feta auf eine Hälfte legen.
  Die andere Hälfte darüber klappen und zusammendrücken.

  \ing[2]{EL}{Rapsöl}

  In einer Grillpfanne mit Öl die Bifteki von allen Seiten braten.

  \freeform{}\hrulefill{}

  \freeform{}\faIcon{lightbulb}
  Mit Tsatziki und Tomatenreis servieren.

  \freeform{}\faIcon{info}
  Nährwerte pro Portion: 273kcal, 8g Kohlenhydrate, 19g Fett, 19g Eiweiß

\end{recipe}
\end{document}
