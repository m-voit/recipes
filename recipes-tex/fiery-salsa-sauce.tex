\documentclass[../recipe-collections/cooking.tex]{subfiles}

\begin{document}
\begin{recipe}{Feurige-Salsa-Sauce}{500ml \faIcon{utensils}}{12h 30min \faIcon{user-clock}, 30min \faIcon{user-cog}}

  \freeform{}\textit{Ergibt eine feurige Sauce.}

  \ing[1]{x}{Zwiebel}
  \ing[2]{x}{Knoblauchzehe}
  \ing[2]{x}{Chilischoten (Jalapenos)}
  \ing[1]{EL}{Olivenöl}

  \textbf{30min}
  Zuerst Zwiebel und Knoblauchzehen schälen und fein würfeln.
  Kerne der Chilischoten entfernen und Schoten würfeln.
  Olivenöl in einer beschichteten Pfanne erhitzen und die Zwiebeln, sowie den Knoblauch glasig dünsten.

  \ing[425]{ml}{Tomaten, stückig}
  \ing[4]{EL}{Apfelessig}
  \ing[2]{EL}{Rohrzucker braun}

  Anschließend Chilischoten, stückige Tomaten, Apfelessig und Rohrzucker in die Pfanne geben.

  \ing[]{}{Pfeffer schwarz}
  \ing[]{}{Meersalz}

  \textbf{30min}
  Die Sauce 30 Minuten bei schwacher Hitze köcheln lassen.
  Gelegentlich umrühren und mit Salz und Pfeffer, sowie Apfelessig abschmecken.

  \newstep{}\textbf{12h}
  Heiß in eine saubere Flasche füllen und anschließend über Nacht ziehen lassen.

  \freeform{}\hrulefill{}

  \freeform{}\faIcon{lightbulb}
  Geschmack kann durch die Verwendung von unterschiedlichen Chilis variiert werden.

  \freeform{}\faIcon{burn}
  Hitzestufe 6–7/9 zum Anbraten und 4/9 zum köcheln lassen.

  \freeform{}\faIcon{snowflake}
  Hält sich im Kühlschrank mehrere Wochen.

\end{recipe}
\end{document}
