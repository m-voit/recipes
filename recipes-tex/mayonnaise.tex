\documentclass[../recipe-collections/cooking.tex]{subfiles}

\begin{document}
\begin{recipe}{Mayonnaise}{\fr34 Glas \faIcon{utensils}}{5min \faIcon{user-cog}}
  \freeform{}\textit{Schnelle Mayonnaise.}

  \ing[200]{ml}{Rapsöl}
  \ing[1]{EL}{Zitronensaft}
  \ing[1]{}{Ei}
  \ing[2]{EL}{Senf mittelscharf}
  \ing[\fr12]{TL}{Salz}
  \ing[1]{TL}{Zucker}
  \ing[\fr12]{TL}{Pfeffer, weiß}

  Es wird ein Pürierstab und ein zylindrisches Gefäß benötigt, das im Durchmesser nur wenig größer sein darf als die Schneide des Pürierstabes.
  Das Rapsöl, den Zitronensaft, das Ei, den Senf, das Salz, den Pfeffer und den Zucker in das Gefäß geben.

  \newstep{}Den Pürierstab bis auf den Boden des Gefäßes absenken, auf voller Stufe einschalten und langsam nach oben ziehen.

  \newstep{}Nach Bedarf mit Kräutern, Petersilie, Dill, Schnittlauch oder Knoblauch verfeinern.

  \freeform{}\hrulefill{}

\end{recipe}
\end{document}
