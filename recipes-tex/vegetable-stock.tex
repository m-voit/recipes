\documentclass[../recipe-collections/cooking.tex]{subfiles}

\begin{document}
\begin{recipe}{Gemüsebrühe}{1l \faIcon{utensils}}{5h \faIcon{user-clock}, 20min \faIcon{user-cog}}

  \freeform{}\textit{Gemüsebrühe als Grundstock für weitere Gerichte.}

  \ing[1]{x}{Zwiebel groß}
  \ing[5-6]{x}{Karotten}
  \ing[]{}{Fett hitzebeständig}
  \ing[1\fr12]{l}{Wasser}

  \textbf{15min}
  Zwiebel und Karotten halbieren und in einem großen unbeschichteten Topf mit sehr hitzebeständigem Fett, wie Biskin, auf einer Seite fast schwarz rösten.
  In mindestens 3 Schritten, nach und nach mit Wasser ablöschen.
  Dabei beim ersten Mal sehr wenig Wasser verwenden.

  \ing[\fr13]{x}{Sellerieknolle}
  \ing[1-2]{Stangen}{Staudensellerie}
  \ing[3]{cm}{Petersilienwurzel}
  \ing[1]{x}{Lauch}
  \ing[4]{cm}{Ingwer}
  \ing[1]{x}{Lorbeerblatt}
  \ing[8]{x}{Lorbeeren}
  \ing[1]{Prise}{Piment}

  \textbf{5h}
  Anschließend alle restlichen Zutaten dazugeben und \textit{mit Deckel} 5h köcheln lassen.
  Die Suppe danach durch ein Sieb abgießen und entweder würzen oder weiterverarbeiten.
  Zum Schluss das ausgekochte Gemüse entfernen und entsorgen.

  \ing[5]{EL}{Sojasauce}
  \ing[1]{Bund}{Petersilie}
  \ing[]{}{Liebstöckel}
  \ing[]{}{Schnittlauch}
  \ing[]{}{Salz}
  \ing[]{}{Pfeffer}

  \textbf{5min}
  Falls die Suppe sofort genossen werden soll, am Ende des Kochens würzen.
  Ansonsten die Gewürze erst zum Ende der Kochzeit des Folgegerichts hinzugeben.

  \freeform{}\hrulefill{}

  \freeform{}\faIcon{lightbulb}
  Ein Bund Petersilie entspricht ca. 30g gefriergetrockneter Kräuter.

\end{recipe}
\end{document}
