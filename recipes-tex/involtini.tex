\documentclass[../recipe-collections/cooking.tex]{subfiles}

\begin{document}
\begin{recipe}{Involtini}{4 Portionen \faIcon{utensils}}{35min \faIcon{user-clock}, 30min \faIcon{user-cog}}

  \freeform{}\textit{Italienische Kalbsrouladen angelehnt an die sizilianische Spezialität Involtini alla Siciliana.}

  \ing[4]{x}{Kalbsschnitzel, 150g}
  \ing[4]{Pr.}{Salz}
  \ing[4]{Pr.}{Pfeffer}
  \ing[4]{Pr.}{Fleischgewürz}

  Die 4 Schnitzel zwischen Frischhaltefolie legen und ca. \fr12 cm flach klopfen.
  Mit je 1 Prise Salz, Pfeffer und Gewürz würzen.

  \ing[4]{Scheiben}{Parmaschinken}
  \ing[50]{g}{Parmesan, gerieben}
  \ing[12]{x}{Salbeiblätter}

  Den Parmaschinken auf einem Brett ausbreiten, die Kalbsschnitzel der Länge nach auf jeweils eine Scheibe legen, sowie mit Parmesan und 3 Salbeiblättern belegen.
  Den Schinken vom kurzen Ende eng um das Fleisch aufrollen.
  Dabei darauf achten, dass die Involtini gut verschlossen sind.

  \ing[2]{EL}{Olivenöl}
  \ing[100]{ml}{Rotwein}
  \ing[400]{ml}{Geflügelbrühe}
  \ing[300]{g}{Balsamico-Zwiebeln}

  Das Olivenöl in der Pfanne erhitzen und die Involtini mit der offenen Kante des Schinkens nach unten in die Pfanne geben.
  Bei starker Hitze je 2 Minuten von jeder Seite anbraten und mit dem Rotwein ablöschen.
  Die Geflügelbrühe sowie die Zwiebeln mit dem Balsamico dazugeben und abgedeckt bei leichter Hitze 35 Minuten köcheln lassen.

  \ing[1]{TL}{Speisestärke}
  \ing[1]{EL}{Wasser}

  Das Fleisch herausnehmen, beiseitestellen und die Sauce bei mittlerer Hitze weiterköcheln lassen.
  Die Speisestärke mit kaltem Wasser verrühren, unter Rühren zur Sauce geben und 1 Minute einkochen lassen.
  Die Involtini wieder in die Sauce geben und 2 Minuten bei leichter Hitze köcheln lassen.

  \newstep{}Die Involtini auf Tellern verteilen und mit der Sauce servieren.

  \freeform{}\hrulefill{}

  \freeform{}\faIcon{lightbulb}
  Mit Salat und frisch gebackenem Brot servieren. Im sizilianischen Original wird Vino de Marsala statt Rotwein verwendet.

\end{recipe}
\end{document}
