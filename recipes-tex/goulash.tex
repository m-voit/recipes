\documentclass[../recipe-collections/cooking.tex]{subfiles}

\begin{document}
\begin{recipe}{Gulasch}{4 Portionen \faIcon{utensils}}{2h 45min \faIcon{user-clock}, 40min \faIcon{user-cog}}

  \freeform{}\textit{Ungarisches Gulasch / Pörkölt.}

  \ing[1000]{g}{Gulasch vom Schwein/Rind}
  \ing[4]{}{Zwiebeln}
  \ing[10]{}{Knoblauchzehen}
  \ing[]{}{Salz}
  \ing[]{}{Zitronensaft, Konzentrat}

  \textbf{60min}
  Zwiebeln und Knoblauch klein hacken und in eine Pfanne geben.
  Das Gulasch in mundgerechte Stücke schneiden (besser zu klein als zu groß) und mit in die Pfanne geben.
  Das Gulasch salzen.
  Jetzt mit den Zwiebeln und Knoblauch vermischen, mit ein paar Tropfen Zitronensaftkonzentrat beträufeln, Deckel darauf und ca. 60 Minuten ziehen lassen.

  \ing[]{}{Sonnenblumenöl}
  \ing[2]{TL}{Brühe, instant}
  \ing[4]{Prisen}{Kümmelpulver}

  \textbf{1h 5min}
  Nach 60 Minuten einen guten Schuss Sonnenblumenöl, kein Olivenöl, in die Pfanne geben und das Gulasch ca. 10 Minuten bei höchster Hitze anbraten.
  Danach bei niedrigerer Hitze ca. 15 Minuten weiter braten.
  Jetzt 500 ml Wasser in die Pfanne geben, eine Prise Kümmel und einen TL Brühe zugeben, dann alles gut verrühren.
  Bei gleichbleibender Hitze unter gelegentlichem Umrühren ca. 40 Minuten ohne Deckel köcheln lassen, bis das Wasser vollkommen verkocht ist.

  \ing[300]{g}{Tomatenmark}
  \ing[3]{TL}{Paprikapulver, rosenscharf}
  \ing[2]{Prisen}{Pfeffer}
  \ing[2]{Prisen}{Majoran}

  \textbf{30min}
  Nun 500 ml Wasser in einem Messbecher mit ca. 100 g Tomatenmark, 1 TL Paprika edelsüß, einer Prise Paprika rosenscharf, Majoran, Pfeffer und noch etwas Kümmel würzen und in die Pfanne geben.
  Bei niedrigerer Hitze nochmal ca. 30 Minuten mit Deckel bei gelegentlichem Umrühren köcheln lassen.

  \freeform{}\hrulefill{}

  \freeform{}\faIcon{lightbulb}
  Als Beilage nimmt man idealerweise Nockerln oder auch kleine Frischei-Spiralnudeln.

  \freeform{}\faIcon{burn}
  Hitzestufe 9/9 zum Anbraten, 6/9 zum Braten und 3/9 zum Köcheln lassen.

\end{recipe}
\end{document}
