\documentclass[../recipe-collections/cooking.tex]{subfiles}

\begin{document}
\begin{recipe}{Piccata Milanese}{4 Portionen \faIcon{utensils}}{5min \faIcon{user-clock}, 25min \faIcon{user-cog}}

  \freeform{}\textit{Auf Basis des Mailänder Original Rezepts.}

  \ing[3]{x}{Rispentomaten}
  \ing[2]{x}{Knoblauchzehen}
  \ing[1]{x}{Zwiebel, rot}

  \textit{Tomaten-Chutney}
  Die Rispentomaten waschen und grob würfeln.
  Den Knoblauch schälen und fein hacken.
  Die Zwiebel schälen, halbieren und in feine Würfel schneiden.

  \ing[1]{EL}{Olivenöl}
  \ing[1]{EL}{Honig}
  \ing[1]{EL}{Balsamicoessig, dunkel}
  \ing[400]{g}{Dosentomaten, stückig}
  \ing[3]{Pr.}{Tomatensauce Gewürz}

  In einem Topf das Olivenöl erhitzen, die Zwiebel und den Knoblauch 3 Minuten bei mittlerer Hitze braten.
  Den Honig hinzufügen und weitere 2 Minuten braten.
  Die Rispentomaten hinzufügen und 2 Minuten braten.
  Mit Balsamicoessig ablöschen, die stückigen Tomaten, sowie das Gewürz hinzugeben und 20 Minuten bei leichter Hitze köcheln lassen.

  \ing[2]{x}{Hähnchenbrustfilets, 300g}
  \ing[4]{Pr.}{Salz}
  \ing[4]{Pr.}{Fleischgewürz}

  \textit{Piccata}
  Die Hähnchenbrustfilets von Haut und Sehnen befreien und waagrecht halbieren.
  Die 4 entstandenen Schnitzel zwischen Frischhaltefolien legen und ca. \fr12 cm flach klopfen.
  Die Schnitzel mit je 1 Prise Salz und Gewürz würzen.

  \ing[5]{x}{Eier}
  \ing[150]{g}{Parmesan, gerieben}
  \ing[100]{g}{Mehl}

  Die Eier in einer Schüssel aufschlagen, den Parmesan dazugeben und gründlich verquirlen.
  Das Mehl auf einen flachen Teller geben.

  \ing[100]{ml}{Olivenöl}

  In 2 Pfannen jeweils 50ml Olivenöl erhitzen.
  Die Schnitzel zuerst im Mehl wenden, dann in die Ei-Parmesan-Mischung geben.
  Dabei darauf achten, dass jede Stelle mit der Panade bedeckt ist.
  Die Schnitzel in die vorgeheizten Pfannen geben und von jeder Seite bei mittlerer Hitze 5 Minuten ausbacken.

  \newstep{}Das Tomaten-Chutney auf einen großen Teller anrichten und die Schnitzel darauf servieren.

  \freeform{}\hrulefill{}

  \freeform{}\faIcon{lightbulb}
  Im Mailänder Original wird Kalbfleisch verwendet.

\end{recipe}
\end{document}
