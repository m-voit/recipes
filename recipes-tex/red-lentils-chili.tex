\documentclass[../recipe-collections/cooking.tex]{subfiles}

\begin{document}
\begin{recipe}{Rotes Linsen-Chili}{4 Portionen \faIcon{utensils}}{15min \faIcon{user-clock}, 15min \faIcon{user-cog}}

  \freeform{}\textit{Deftiger Eintopf für die kalte Jahreszeit.}

  \ing[2]{x}{Zwiebeln, rot}
  \ing[1]{x}{Knoblauchzehe}
  \ing[1]{x}{Paprika, rot}
  \ing[2]{EL}{Öl}

  Die Zwiebeln und den Knoblauch pellen, die Paprika entkernen und alles in feine Würfel schneiden.
  In einem hohen Topf mit etwas Öl bei mittlerer Hitze anschwitzen.

  \ing[250]{g}{Linsen, rot}

  Sobald die Zwiebeln glasig sind, die Linsen hinzugeben und für ca. 2 Minuten mit anschwitzen.

  \ing[350]{ml}{Gemüsebrühe}
  \ing[400]{ml}{Tomaten, gehackt}

  Mit der Gemüsebrühe ablöschen und die gehackten Tomaten hinzugeben.
  Alles für weitere 10 Minuten kochen, bis die Linsen fast gar sind.

  \ing[400]{ml}{Bohnen, weiß}
  \ing[100]{g}{Mais}
  \ing[2]{TL}{Cayennepfeffer}
  \ing[2]{EL}{Naturjoghurt}
  \ing[30]{g}{Lauchzwiebeln}

  Die weißen Bohnen, den Mais sowie den Cayennepfeffer unterrühren, für weitere 2 Minuten köcheln lassen und mit Lauchzwiebeln und Naturjoghurt garniert servieren.

  \freeform{}\hrulefill{}

  \freeform{}\faIcon{info}
  Nährwerte pro Portion: 490kcal, 60g Kohlenhydrate, 13g Fett, 33g Eiweiß

\end{recipe}
\end{document}
