\documentclass[../recipe-collections/cooking.tex]{subfiles}

\begin{document}
\begin{recipe}{Tsatziki}{4 Portionen \faIcon{utensils}}{12h \faIcon{user-clock}, 10min \faIcon{user-cog}}

  \freeform{}\textit{Klassisches griechisches Tsatziki.}

  \ing[\fr12]{x}{Bio-Salatgurke}

  \textbf{5min}
  Gurke waschen und mithilfe einer Reibe grob Raspeln.
  Gurkenraspeln in ein feines Sieb geben und gut abtropfen lassen, evtl. leicht ausdrücken.

  \ing[500]{g}{griechischer Joghurt}
  \ing[1]{x}{Knoblauchzehe}
  \ing[2]{EL}{Olivenöl}
  \ing[1]{Prise}{Salz}
  \ing[1]{Prise}{Pfeffer}

  \textbf{5min}
  Joghurt in eine Schale geben und glatt rühren.
  Knoblauch schälen und fein hacken (oder pressen), in den Joghurt geben.
  Olivenöl und Gurkenraspeln dazu geben, alles glattrühren.
  Mit Salz und Pfeffer abschmecken.

  \newstep{}\textbf{12h}
  Mindestens 12h, am besten über Nacht im Kühlschrank ziehen lassen.

  \freeform{}\hrulefill{}

  \freeform{}\faIcon{info}
  Nährwerte pro Portion: 207kcal, 5.97g Kohlenhydrate, 18.6g Fett, 4.16g Eiweiß

\end{recipe}
\end{document}
