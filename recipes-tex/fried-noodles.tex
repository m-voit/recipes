\documentclass[../recipe-collections/cooking.tex]{subfiles}

\begin{document}
\begin{recipe}{Asiatische Bratnudeln}{4 Portionen \faIcon{utensils}}{1h \faIcon{user-clock}, 45min \faIcon{user-cog}}

  \freeform{}\textit{}

  \ing[300]{g}{Hähnchenbrust}
  \ing[]{}{Salz}
  \ing[]{}{Pfeffer}

  \textbf{1h}
  Den Ofen auf 180°C Umluft vorheizen.
  Ein Backblech mit Backpapier auslegen.
  Hähnchenbrustfilet waschen, trocken tupfen, mit Salz und Pfeffer würzen und auf das Backblech legen.
  Im Ofen ca. 45 Minuten backen.

  \ing[250]{g}{Mienudeln}
  \ing[1]{Dose}{Bambussprossen, streifen}
  \ing[1]{Stange}{Lauch}
  \ing[1]{x}{Paprikaschote, rot}
  \ing[250]{g}{Pak-Choi}
  \ing[3]{EL}{Honig}
  \ing[1]{EL}{Sesam}

  \textbf{30min}
  Die Mie Nudeln nach Packungsanleitung zubereiten.
  Die Sprossen abtropfen lassen.
  Den Lauch putzen, waschen und in Ringe schneiden.
  Die rote Paprika putzen, waschen und in Streifen schneiden.
  Den Pak Choi putzen, waschen und in kleine Stücke schneiden.
  Den Sesam und Honig verrühren und die Hähnchenbrust damit einstreichen.
  Kurz ziehen lassen und dann in Scheiben schneiden.

  \ing[2]{EL}{Öl}
  \ing[6]{EL}{Sojasauce o. Glutamat}
  \ing[1]{EL}{Sriracha}
  \ing[\fr14]{TL}{Habanero Chili}
  \ing[1]{TL}{Schwarzkümmel}

  \textbf{15min}
  Das Öl in einer großen Pfanne erhitzen.
  Das Gemüse ca. 3 Minuten darin braten.
  Mit Sojasoße, Habanero Chili und Sriracha würzen.
  Die Mie-Nudeln untermischen und mitbraten.
  Die Bratnudeln und das Fleisch anrichten und mit Schwarzkümmel garnieren.

  \freeform{}\hrulefill{}

  \freeform{}\faIcon{lightbulb}
  Lang Tzu Gewürz über das Gericht streuen.

\end{recipe}
\end{document}
