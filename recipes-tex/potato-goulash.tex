\documentclass[../recipe-collections/cooking.tex]{subfiles}

\begin{document}
\begin{recipe}{Kartoffelgulasch}{4 Portionen \faIcon{utensils}}{??min \faIcon{user-clock}, ??min \faIcon{user-cog}}

  \freeform{}\textit{Deftiger Eintopf ohne Fleisch.}

  \ing[500]{g}{Kartoffeln}
  \ing[1]{x}{Zwiebel}
  \ing[2]{x}{Paprikaschoten}

  Die Kartoffeln waschen, schälen und in große Würfel schneiden.
  Nun die Zwiebel putzen, waschen, halbieren und in Scheiben schneiden.
  Anschließend die Paprikaschote würfeln.

  \ing[3]{EL}{Öl}

  Öl im Dampfdrucktopf erhitzen, die Zwiebeln zugeben und anbräunen.
  Dann die Kartoffeln zugeben, von allen Seiten braten und dabei gut umrühren.

  \ing[1]{Dose}{gehackte Tomaten}
  \ing[1]{TL}{gekörnte Brühe}
  \ing[\fr12]{TL}{Salz}
  \ing[1]{Pr.}{Pfeffer}
  \ing[1]{Pr.}{Paprikapulver}
  \ing[1]{Pr.}{Chili}
  \ing[1]{Pr.}{Majoran}
  \ing[2]{EL}{Tomatenmark}
  \ing[1]{x}{Knoblauchzehe}
  \ing[]{}{Wasser}

  Die gehackten Tomaten, die gekörnte Brühe und die Gewürze zugeben.
  Anschließend das Tomatenmark hinzugeben.
  Nun die Knoblauchzehe schälen und dazu pressen.
  Das Gulasch gut aufrühren, bei Bedarf mit etwas Wasser aufgießen und abschmecken.

  \newstep{}Den Dampfdrucktopf nach Vorschrift bedienen.
  Die Garzeit beträgt aber dem Erscheinen des Druckanzeigers noch 5 Minuten.

  \ing[\fr12]{Becher}{Schmand}
  \ing[]{}{Petersilie}

  Nach der Garzeit, den Schmand zugeben, das Kartoffelgulasch erneut abschmecken und anrichten.
  Abschließend mit fein gehackter Petersilie garnieren.

  \freeform{}\hrulefill{}

  \freeform{}\faIcon{lightbulb}
  Es wird ein Dampfdrucktopf benötigt.

\end{recipe}
\end{document}
