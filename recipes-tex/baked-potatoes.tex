\documentclass[../recipe-collections/cooking.tex]{subfiles}

\begin{document}
\begin{recipe}{Ofenkartoffeln}{6 Personen \faIcon{utensils}}{1h \faIcon{user-clock}, 30min \faIcon{user-cog}}

  \freeform{}\textit{Patate arossto.}

  \ing[1]{kg}{Kartoffeln}
  \ing[3]{x}{Knoblauchzehen}
  \ing[1]{x}{Rosmarinzweig}
  \ing[5-6]{x}{Salbeiblätter}
  \ing[]{}{Salz}
  \ing[]{}{Pfeffer}
  \ing[8]{EL}{Olivenöl Extravergine}

  \textbf{30min}
  Die Kartoffeln schälen, waschen und in große Würfel schneiden.
  Mit einem Geschirrtuch trocken tupfen und in eine große, mit 3–4 Esslöffeln Olivenöl beträufelten Ofenform geben.
  Wenn die Kartoffeln nicht übereinander liegen, werden sie knuspriger.
  Mit Salz und Pfeffer würzen, den Knoblauch im Hemd (ungeschält und leicht zerdrückt), die Salbeiblätter, den in 2–3 Stücke zerteilten Rosmarinzweig hinzufügen und mit dem übrig gebliebenen Öl begießen.

  \newstep{}\textbf{1h}
  Die Kartoffeln im auf 200 °C vorgeheizten Ofen für 40–50 Minuten backen.
  Die Form etwa alle 10 Minuten herausnehmen und die Kartoffeln vorsichtig, um sie nicht zu zerdrücken, mit einem Holzlöffel wenden.
  Die Knoblauchzehen herausnehmen, sobald sie goldbraun sind.
  Sie dürfen nicht verbrennen, um den Geschmack der Kartoffeln nicht zu beeinträchtigen.
  Heiß als Beilage oder Zwischengericht mit Gemüse servieren.

  \freeform{}\hrulefill{}

  \freeform{}\faIcon{burn}
  Den Backofen auf 200 °C vorheizen.

\end{recipe}
\end{document}
