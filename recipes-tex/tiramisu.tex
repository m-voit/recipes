\documentclass[../recipe-collections/cooking.tex]{subfiles}

\begin{document}
\begin{recipe}{Tiramisu}{6 Personen \faIcon{utensils}}{4h \faIcon{user-clock}, 30min \faIcon{user-cog}}

  \freeform{}\textit{Burgis Tiramisu.}

  \ing[4]{}{Eigeln}
  \ing[4]{EL}{Zucker}
  \ing[500]{g}{Mascarpone}
  \ing[2]{EL}{Eierlikör}
  \ing[3]{}{Eiweiß}
  \ing[]{}{Biskuit}
  \ing[\fr14]{Liter}{Kaffee}
  \ing[]{}{Amaretto}

  Eigelb mit Zucker schaumig schlagen.
  Mascarpone und Eierlikör gut unterrühren.
  Den geschlagenen Eischnee unterheben.

  \newstep{}Boden mit Biskuit auslegen und mit Kaffee tränken und mit Amaretto beträufeln.

  \newstep{}1/3 der Creme darauf streichen, zweite Lage Biskuit tränken usw.
  Letzte Lage ist Creme und mit Kakao betreuen.

  \newstep{}\textbf{4h}
  Tiramisu ruhen lassen.

  \freeform{}\hrulefill{}

  \freeform{}\textit{Variante Erdbeertiramisu.}

  \ing[]{}{Erdbeeren}

  Creme ohne Eierlikör machen.
  Biskuit mit pürierten Erdbeeren und Zucker tränken.
  Erdbeerstückchen in die Creme einstreuen.

  \freeform{}\hrulefill{}

\end{recipe}
\end{document}
