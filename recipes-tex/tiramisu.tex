\documentclass[../recipe-collections/cooking.tex]{subfiles}

\begin{document}
\begin{recipe}{Tiramisu}{6 Personen \faIcon{utensils}}{8h \faIcon{user-clock}, 45min \faIcon{user-cog}}


  \freeform{}\textit{Tiramisu nach italienischer Art}

  \ing[150]{ml}{Espresso}

  \faIcon{snowflake}
  Den Espresso kochen und kalt stellen.
  Dieser muss für das Gelingen des Tiramisus eiskalt sein!


  \ing[4]{\X}{Eigelb}
  \ing[4]{\X}{Eiweiß}

  Die Eiweiße und Eigelbe in getrennte Schüsseln geben.
  Dabei darauf achten, dass die Eiweiße nicht mit Eigelb vermischt werden!

  \ing[50]{g}{Puderzucker}
  \ing[2]{Pr.}{Salz}

  Zuerst die Eiweiße mit der Hälfte des Puderzuckers und 1 Prise Salz in eine Schüssel geben und mit einem Handmixer steif schlagen.
  Anschließend die Eigelbe zusammen mit dem verbliebenen Puderzucker und 1 Prise Salz in eine Schüssel geben und mit einem Handmixer schaumig schlagen.

  \ing[500]{g}{Mascarpone}

  Nun den Mascarpone jeweils zur Hälfte in die Schüsseln geben, kurz glatt rühren und dabei darauf achten, dass der Mascarpone nicht flockig wird.
  Auch hier die Eiweiße und Eigelbe nicht vermischen!

  \ing[50]{ml}{Amaretto}
  \ing[250]{g}{Löffelbiskuit}

  Den kalten Espresso mit dem Amaretto verrühren.
  Die Löffelbiskuits kurz von jeder Seite in die Espressomischung tauchen und insgesamt maximal 3s eintauchen.
  Mit der Zuckerseite nach unten gedrehten Löffelbiskuits den Boden einer Form eng auslegen, die etwa 12 Biskuits je Lage fasst.
  Die erste Lage mit der Creme aus den Eigelben bestreichen.

  \ing[50]{g}{Kakaopulver}
  \ing[]{}{Zartbitterschokolade}

  Die erste Schicht mit Kakaopulver und geriebener Zartbitterschokolade bestreuen.

  \newstep{}Nun die zweite Schicht getränkte Löffelbiskuits darüberlegen und mit der Creme aus den Eiweißen bestreichen.
  Auch diese mit dem verbliebenen Kakaopulver und der geriebenen Zartbitterschokolade bestreuen.

  \newstep{}
  \faIcon{snowflake}
  \textbf{8h+}
  Über Nacht gut abgedeckt kalt stellen und durchziehen lassen.
  Wenn nicht anders möglich mindestens 2h kalt stellen.

  \freeform\hrulefill

  \freeform{}\faIcon{lightbulb}
  Schokoloade kühl stellen, damit sie sich besser reiben lässt.

  \newpage{}

  \freeform\hrulefill
  \freeform{}\textit{Tiramisu von Burgi}

  \ing[4]{\X}{Eigelb}
  \ing[4]{EL}{Zucker}
  \ing[500]{g}{Mascarpone}
  \ing[2]{EL}{Eierlikör}
  \ing[3]{\X}{Eiweiß}
  \ing[]{}{Löffelbiskuit}

  Das Eigelb mit Zucker schaumig schlagen.
  Die Mascarpone und Eierlikör gut unterrühren.
  Nun Den geschlagenen Eischnee unterheben.

  \ing[\fr14]{Liter}{Kaffee}
  \ing[]{}{Amaretto}

  Den Boden mit Biskuit auslegen und mit Kaffee tränken und mit Amaretto beträufeln.

  \newstep{}Nun \fr13 der Creme darauf streichen, die zweite Lage mit Biskuit tränken usw.
  Die letzte Lage ist Creme, die abschließend mit Kakao betreuen.

  \freeform{}\hrulefill{}

  \freeform{}\textit{Variante Erdbeertiramisu von Burgi}

  \ing[]{}{Erdbeeren}

  Die Creme ohne Eierlikör zubereiten.
  Biskuit mit pürierten Erdbeeren und Zucker tränken.
  Erdbeerstückchen in die Creme einstreuen.

  \freeform{}\hrulefill{}

\end{recipe}
\end{document}
