\documentclass[../recipe-collections/cooking.tex]{subfiles}

\begin{document}
\begin{recipe}{Tiramisu}{6 Personen \faIcon{utensils}}{4h \faIcon{user-clock}, 30min \faIcon{user-cog}}


  \freeform{}\textit{Tiramisu italienisch}

  \ing[5]{}{Eigelb}
  \ing[5]{}{Eiweiß}
  \ing[70]{g}{Puderzucker}
  \ing[750]{g}{Mascarpone}
  \ing[1]{Pr.}{Salz}

  Die Eigelbe zusammen mit dem Puderzucker und 1 Prise Salz in eine Schüssel geben und mit einem Handmixer 10 Minuten schaumig schlagen.
  Den Mascarpone dazugeben, kurz glatt rühren und dabei darauf achten, dass der Mascarpone nicht flockig wird.

  \ing[250]{ml}{Espresso}
  \ing[50]{ml}{Amaretto}
  \ing[250]{g}{Löffelbiskuit}

  Den kalten Espresso mit dem Amaretto verrühren.
  Die Löffelbiskuits kurz von jeder Seite in die Espressomischung tauchen und den Boden einer Form (ca. 23 x 33cm) eng damit auslegen.
  Die Hälfte der Mascarponecreme auf den Löffelbiskuits in der Form verteilen.
  Erneut eine Schicht getränkte Löffelbiskuits darüberlegen und mit der restlichen Mascarponecreme bestreichen.

  \newstep{}1/3 der Creme darauf streichen, zweite Lage Biskuit tränken usw.
  Letzte Lage ist Creme und mit Kakao betreuen.

  \newstep{}\textbf{6h+}
  Mindestens 6 Stunden oder am Besten über Nacht gut abgedeckt kalt stellen und durchziehen lassen.

  \ing[50]{g}{Kakaopulver}
  \ing[]{}{Zartbitterschokolade}
  Kurz vor dem Servieren großzügig mit Kakaopulver bestreuen.
  Nach Geschmack zusätzlich mit geriebener Zartbitterschokolade bestreuen.

  \freeform{}\textit{Tiramisu von Burgi}

  \ing[4]{}{Eigelb}
  \ing[4]{EL}{Zucker}
  \ing[500]{g}{Mascarpone}
  \ing[2]{EL}{Eierlikör}
  \ing[3]{}{Eiweiß}
  \ing[]{}{Biskuit}
  \ing[\fr14]{Liter}{Kaffee}
  \ing[]{}{Amaretto}

  Eigelb mit Zucker schaumig schlagen.
  Mascarpone und Eierlikör gut unterrühren.
  Den geschlagenen Eischnee unterheben.

  \newstep{}Boden mit Biskuit auslegen und mit Kaffee tränken und mit Amaretto beträufeln.

  \newstep{}1/3 der Creme darauf streichen, zweite Lage Biskuit tränken usw.
  Letzte Lage ist Creme und mit Kakao betreuen.

  \newstep{}\textbf{4h}
  Tiramisu ruhen lassen.

  \freeform{}\hrulefill{}

  \freeform{}\textit{Variante Erdbeertiramisu von Burgi}

  \ing[]{}{Erdbeeren}

  Creme ohne Eierlikör machen.
  Biskuit mit pürierten Erdbeeren und Zucker tränken.
  Erdbeerstückchen in die Creme einstreuen.

  \freeform{}\hrulefill{}

\end{recipe}
\end{document}
