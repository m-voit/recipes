\documentclass[../recipe-collections/cooking.tex]{subfiles}

\begin{document}
\begin{recipe}{Linseneintopf}{4 Portionen \faIcon{utensils}}{40min \faIcon{user-clock}, 20min \faIcon{user-cog}}

  \freeform{}\textit{Deftiger Eintopf für die kalte Jahreszeit.}

  \ing[200]{g}{Linsen}
  \ing[\fr12]{l}{Gemüsebrühe}
  \ing[2]{x}{Kartoffeln festkochend, klein}
  \ing[1]{x}{Karotte, klein}
  \ing[1]{x}{Zwiebel, groß}
  \ing[100]{g}{Cocktailtomaten}

  Die Linsen mit der Brühe aufkochen und zugedeckt in 40 Minuten fast weich garen.
  Kartoffeln und Karotte schälen, waschen und würfeln.
  Die Zwiebel schälen und hacken.
  Die Tomaten abziehen.

  \ing[3]{EL}{Öl}

  Das Öl erhitzen und die Zwiebel darin glasig dünsten.
  Kartoffeln und Karotte zugeben und unter Rühren anbraten.
  Zu den Linsen geben und zugedeckt bei schwacher Hitze in 5 bis 10 Minuten weich garen.

  \ing[200]{g}{Schmant}
  \ing[1]{EL}{Schnittlauch}
  \ing[]{}{Salz}
  \ing[]{}{Pfeffer, weiß}
  \ing[2\fr12]{EL}{Sojasauce}

  Tomaten und Schmant untermischen und erhitzen.
  Mit Salz und Pfeffer würzen und mit Schnittlauch bestreuen.
  Optional mit Sojasauce abschmecken.

  \freeform{}\hrulefill{}

  \freeform{}\faIcon{info}
  Nährwerte pro Portion: 439kcal, 44g Kohlenhydrate, 19g Fett, 16g Eiweiß, 10g Ballaststoffe

\end{recipe}
\end{document}
