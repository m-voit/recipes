\documentclass[../recipe-collections/cooking.tex]{subfiles}

\begin{document}
\begin{recipe}{Gegrillte Süßkartoffeln}{4 Portionen \faIcon{utensils}}{15min \faIcon{user-clock}, 1h \faIcon{user-cog}}

  \freeform{}\textit{In Scheiben geschnittene und marinierte Süßkartoffeln als Grillbeilage.}

  \ing[750]{g}{Süßkartoffeln}

  Die Süßkartoffeln waschen und in \fr12 cm dicke Scheiben schneiden.

  \ing[3]{x}{Knoblauchzehen}

  Die Knoblauchzehen schälen und fein hacken.

  \ing[7]{EL}{Olivenöl}
  \ing[2]{TL}{Thymian}
  \ing[2]{TL}{Rosmarin}
  \ing[2]{TL}{Estragon}
  \ing[1]{TL}{Chilipulver}

  Die Gewürze mit dem Olivenöl mischen und die Knoblauchzehen hinzugeben.

  \ing[]{}{Salz}
  \ing[]{}{Pfeffer}

  Die Marinade mit Salz und Pfeffer abschmecken.

  \newstep{}\textbf{1h}
  Die Süßkartoffeln etwa 1h vor dem Grillen mit der Marinade bestreichen und ziehen lassen.
  Dabei etwas Marinade übrig lassen und die gegrillten Süßkartoffeln nach dem Grillen erneut mit Marinade einstreichen.

  \freeform{}\hrulefill{}

  \freeform{}\faIcon{lightbulb}
  Die Süßkartoffeln sind fertig, wenn sie weich werden. Um die Grillzeit zu verkürzen, kann die Süßkartoffel vorgekocht werden.

  \freeform{}\faIcon{burn}
  Indirekte Hitze bei 250°C.

\end{recipe}
\end{document}
