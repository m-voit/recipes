\documentclass[../recipe-collections/cooking.tex]{subfiles}

\begin{document}
\begin{recipe}{BBQ-Sauce mit Kellerbier} {500ml \faIcon{utensils}} {13h \faIcon{user-clock}, 30min \faIcon{user-cog}}

  \freeform{}\textit{Ergibt eine scharf würzige BBQ-Sauce mit malziger Bier Note.}

  \ing[1]{x}{Zwiebel rot, mittelgroß}
  \ing[2]{x}{Knoblauchzehe}
  \ing[4]{EL}{Olivenöl}
  \ing[340]{ml}{Kellerbier süffig}

  \textbf{20min}
  Zunächst Zwiebel und Knoblauch abziehen und in feine Würfel schneiden.
  Zwiebel und Knoblauch im erhitzten Öl glasig anschwitzen.
  Das Ganze dann mit dem Kellerbier ablöschen.

  \ing[400]{ml}{Tomaten passiert}
  \ing[8]{EL}{Honig}
  \ing[4]{EL}{Worcester Sauce}
  \ing[100]{g}{Tomatenmark}
  \ing[2]{TL}{Dijon-Senf}

  \textbf{5min}
  Anschließend die passierten Tomaten, den Honig, die Worcester Sauce, das Tomatenmark und den Senf hinzufügen.

  \ing[1]{TL}{Salz}
  \ing[1]{TL}{Pfeffer}
  \ing[\fr12]{TL}{Paprikapulver geräuchert}
  \ing[\fr12]{TL}{Chipotle Chili}

  \textbf{5min}
  Den Pfeffer, die Chipotle Chili, die geräucherte Paprika und das Salz hinzugeben.
  Dann kurz aufkochen lassen.

  \newstep{}\textbf{60min}
  Bei geringer Hitze die Sauce etwa 60 Minuten köcheln lassen.
  Zum Abschluss die Sauce abschmecken und nach Bedarf nachwürzen.

  \newstep{}\textbf{12h}
  Anschließend über Nacht ziehen lassen.

  \freeform{}\hrulefill{}

  \freeform{}\faIcon{lightbulb}
  Dünnflüssig eingekocht als Grillglasur verwenden.

  \freeform{}\faIcon{burn}
  Hitzestufe 6–7/9 zum Anbraten und 4/9 zum köcheln lassen.

  \freeform{}\faIcon{snowflake}
  Hält sich im Kühlschrank mehrere Wochen.

\end{recipe}
\end{document}
