\documentclass[../recipe-collections/cooking.tex]{subfiles}

\begin{document}
\begin{recipe}{Kichererbseneintopf}{4 Portionen \faIcon{utensils}}{45min \faIcon{user-clock}, 15min \faIcon{user-cog}}

  \freeform{}\textit{Kichererbseneintopf mit Kartoffeln und Maronen}

  \ing[200]{g}{Kartoffeln, festkochend}
  \ing[1]{x}{Zwiebel}
  \ing[2]{x}{Knoblauchzehen}
  \ing[2]{x}{Rispentomaten}
  \ing[300]{g}{Maronen, gekocht, geschält}
  \ing[1]{Bund}{Petersilie}
  \ing[\fr12]{Bund}{Thymian}
  \ing[1200]{g}{Kichererbsen}

  \textbf{15min}
  Die Kartoffeln schälen und in ca. 2 × 2 cm große Würfel schneiden.
  Die Zwiebel schälen, halbieren und in feine Würfel schneiden.
  Den Knoblauch schälen und fein hacken.
  Die Rispentomaten waschen und in grobe Stücke schneiden.
  Die Maronen ebenfalls in grobe Stücke schneiden.
  Die Petersilie und den Thymian waschen, trocken tupfen und fein hacken.
  Die Kichererbsen durch ein Sieb abgießen, abbrausen und abtropfen lassen.

  \ing[1]{EL}{Olivenöl}
  \ing[500]{ml}{Wasser}
  \ing[400]{g}{Tomaten, passiert}

  \textbf{45min}
  Das Olivenöl in einem großen Topf erhitzen und Knoblauch, Zwiebel- und Kartoffelwürfel 5 Minuten bei mittlerer Hitze anbraten.
  Die Rispentomaten und die Maronen dazugeben und mit dem Wasser und den passierten Tomaten ablöschen.
  Die Kichererbsen hinzufügen und 40 Minuten bei leichter Hitze köcheln lassen.
  Dabei gelegentlich umrühren.

  \ing[50]{g}{Parmesan, gerieben}
  \ing[2]{Pr.}{Salz}
  \ing[2]{Pr.}{Pfeffer}
  \ing[5]{Pr.}{Carbonara Gewürz}

  Kurz vor Ende der Garzeit die Petersilie, den Thymian und den Parmesan zum Eintopf geben und mit dem Carbonara Gewürz, 2 Prisen Salz und 2 Prisen Pfeffer würzen.

  \freeform{}\hrulefill{}

  \freeform{}\faIcon{info}
  Nährwerte pro Portion: 618kcal, 87g Kohlenhydrate, 14g Fett, 26g Eiweiß

\end{recipe}
\end{document}
