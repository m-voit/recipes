\documentclass[../recipe-collections/cooking.tex]{subfiles}

\begin{document}
\begin{recipe}{Zwiebelrostbraten}{4 Portionen \faIcon{utensils}}{1h \faIcon{user-cog}}

  \freeform{}\textit{Zwiebelrostbraten.}

  \ing[4]{Scheiben}{Roastbeef}
  \ing[2]{x}{Zwiebeln}
  \ing[1]{x}{Butter, kalt}
  \ing[1]{x}{Öl}

  Die Zwiebeln werden in Ringe geschnitten und dann leicht mit Mehl gestaubt.
  Dann in heißem Öl goldbraun backen.
  Öfters umrühren.
  Aus dem Öl nehmen und auf Küchenpapier abtropfen lassen.

  \ing[1]{x}{Salz}
  \ing[1]{x}{Pfeffer}
  \ing[1]{x}{Mehl}

  Das Fleisch mit Salz und Pfeffer würzen und auf einer Seite mit Mehl bestäuben.
  Dann mit der Mehlseite in heißem Öl anbraten.
  Wenden, kurz braten, herausnehmen und dann warm stellen.

  \ing[100]{ml}{Rotwein}
  \ing[600]{ml}{Rinderbrühe}
  \ing[1]{x}{Maisstärke}

  Den Bratrückstand mit Rotwein ablöschen und dann reduzieren.
  Dann mit der Rindsuppe aufgießen und nochmals reduzieren, bis eine schöne dunkle Sauce entsteht.
  Mit Salz und Pfeffer würzen und mit kalter Butter montieren.
  Bei Bedarf kann man die Sauce noch mit etwas Maisstärke eindicken.

  \newstep{}\textbf{5 min}
  Das Fleisch wieder in die Sauce geben und kurz ziehen lassen.
  Auf Tellern anrichten, mit Sauce begießen und die leicht gesalzenen Zwiebelscheiben darauf platzieren.

  \freeform{}\hrulefill{}

  \freeform{}\faIcon{lightbulb}
  Mit Reis, Salz- oder Bratkartoffeln servieren.

\end{recipe}
\end{document}
