\documentclass[../recipe-collections/cooking.tex]{subfiles}

\begin{document}
\begin{recipe}{Semmeln} {\faIcon{utensils}} {12h 25min \faIcon{user-clock}, 15min \faIcon{user-cog}}

  \freeform{}\textit{Weiße Semmeln.}

  \ing[500]{g}{Dinkelmehl}
  \ing[320]{g}{Wasser}
  \ing[15]{g}{Hefe}
  \ing[15]{g}{Butter oder Öl}
  \ing[1]{TL}{Salz}
  \ing[1]{TL}{Zucker}

  \textbf{15min}
  Verrühren bis der Teig eine Konsistenz wie ein Spätzleteig erreicht.

  \newstep{}\textbf{12h}
  Über Nacht in den Kühlschrank stellen.
  Semmeln abstechen und danach nochmal etwas gehen lassen.

  \newstep{}\textbf{25min}
  Umluft mit Unterbodenhitze bei 225–250° (je nach Ofen).
  25 min backen und am Schluss ein wenig heißer, damit die Semmeln knusprig werden.

  \freeform{}\hrulefill{}

\end{recipe}
\end{document}
