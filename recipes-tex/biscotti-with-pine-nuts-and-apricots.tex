\documentclass[../recipe-collections/cooking.tex]{subfiles}

\begin{document}
\begin{recipe}{Biscotti mit Pinienkernen und Aprikosen} {40-50 Stück \faIcon{utensils}} {45min \faIcon{user-clock}, 30min \faIcon{user-cog}}

  \freeform{}\textit{Auch bekannt als Cantuccini.}

  \ing[150]{g}{Pinienkerne}
  \ing[280]{g}{Weizenmehl, Type 405}
  \ing[\fr12]{TL}{Backpulver}
  \ing[125]{g}{Aprikosen, getrocknet}

  Die Pinienkerne in einer Pfanne ohne Fett hellbraun rösten und abkühlen lassen.
  Das Mehl mit dem Backpulver mischen und in eine Schüssel sieben.
  Die getrockneten Aprikosen fein hacken.

  \ing[125]{g}{Butter, weich}
  \ing[185]{g}{Puderzucker}
  \ing[2]{x}{Eier, Gr.\ M}
  \ing[\fr12]{TL}{Zimtpulver}
  \ing[1]{Pr.}{Salz}
  \ing[1]{TL}{Orangenschale, gerieben}
  \ing[]{}{Mehl}

  Die Butter mit dem Puderzucker in einer Schüssel schaumig schlagen.
  Die Eier nacheinander unterschlagen und den Zimt, 1 Prise Salz und die Orangenschale unterrühren.
  Die Mehlmischung dazugeben, alles gut mischen und zu einem geschmeidigen Teig verkneten.
  Die Aprikosenstücke und die Pinienkerne unter den Teig kneten.


  \newstep{}Den Teig vierteln und mit bemehlten Händen zu 30~cm langen Rollen formen.
  Das Backblech mit Backpapier auslegen.
  Die Teigrollen mit ausreichend Abstand voneinander auf das Blech legen.
  Die Rollen gleichmäßig etwas flach drücken, sodass sie 4 cm breit und 1,5 cm dick sind.

  \newstep{}\textbf{45min}
  Im vorgeheizten Backofen bei 180°C (Ober- und Unterhitze) auf der 2. Einschubebene von unten 30–35 Minuten backen.
  Die Teigrollen aus dem Ofen nehmen und auf einem Kuchengitter leicht abkühlen lassen, den Ofen nicht ausschalten.
  Die Teigrollen in 2~cm dicke Scheiben schneiden und die Scheiben mit etwas Abstand auf das Backblech legen.
  Nochmals etwa 10 Minuten backen, bis sie leicht gebräunt sind.
  Die Biscotti aus dem Ofen nehmen und vollständig auskühlen lassen.

  \freeform{}\hrulefill{}

  \freeform{}\faIcon{burn}
  Bei 180°C (Ober- und Unterhitze) in der 2.\ Einschubebene von unten backen.
  Die Backzeit beträgt 30–45 Minuten plus Vorheizen.

\end{recipe}
\end{document}
