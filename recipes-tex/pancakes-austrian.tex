\documentclass[../recipe-collections/cooking.tex]{subfiles}

\begin{document}
\begin{recipe}{Palatschinken}{4 Portionen \faIcon{utensils}}{30min \faIcon{user-clock}, 20min \faIcon{user-cog}}

  \freeform{}\textit{Pfannkuchen.}

  \ing[250]{ml}{Milch}
  \ing[2]{}{Eier (Größe M)}
  \ing[1]{TL}{Zucker}
  \ing[1]{Prise}{Salz}
  \ing[210]{g}{Mehl}
  \ing[1]{EL}{Butter}

  \textbf{30min}
  In einer Schüssel Milch, Eier, Zucker und Salz verrühren.
  Zuletzt das Mehl untermischen und ca. 30min quellen lassen.
  Eine beschichtete Pfanne dünn mit Butter aus pinseln und heiß werden lassen.
  Temperatur auf mittlere Hitze zurückschalten.

  \newstep{}Danach auf niedriger Stufe bis zur gewünschten Konsistenz erhitzen.

  \newstep{}Ca. 1/8 des Teiges in die Pfanne gießen und durch rasches Hin- und Herschwenken gleichmäßig und dünn verteilen.
  Ca. 1 Minute braten, bis sich die Unterseite vom Pfannenboden lost, wenden und in 1/2 Minute fertig braten.
  Fertige Palatschinken auf einen Teller stapeln und zugedeckt warm halten.
  Übrigen Teig ebenso verarbeiten, dabei vor jedem Palatschinken die Pfanne erneut mit wenig Butter fetten.

  \newstep{}Vor dem Servieren mit Marmelade bestreichen.
  Einrollen oder zusammenklappen und mit Zucker oder Puderzucker bestreuen.

  \freeform{}\hrulefill{}

  \freeform{}\faIcon{lightbulb}
  Beschichtete Pfanne verwenden.

  \newpage{}

  \freeform{}\hrulefill{}

  \freeform{}\textit{Variante Nussfüllung.}

  \ing[200]{g}{Milch}
  \ing[2]{EL}{Honig}
  \ing[50]{g}{Zucker}
  \ing[200]{g}{Haselnüsse gemahlen}
  \ing[3-4]{EL}{Rum}
  \ing[1]{}{Zimtpulver}

  \textbf{10 min}
  Milch mit Honig und Zucker aufkochen lassen, gemahlene Haselnüsse einrühren und mit Rum und Zimtpulver abschmecken.
  Danach ca. 10 Minuten quellen lassen.
  Auf die Palatschinken streichen und wie eine Schnecke fest zusammen rollen.

  \freeform{}\hrulefill{}

  \freeform{}\textit{Variante Apfel-Palatschinken.}

  \ing[175]{g}{Wasser}
  \ing[100]{g}{Zucker}
  \ing[\fr12]{TL}{Zimtpulver}
  \ing[1]{}{Saft von 1 Zitrone}
  \ing[500]{g}{Äpfel}

  In einem Topf Wasser, Zucker, Zimtpulver und den Zitronensaft unter Rühren zu einem dickflüssigen Sirup einkochen lassen.
  Äpfel schälen, Kerngehäuse entfernen, klein würfeln und zum Sirup geben.
  Unter Rühren kochen, bis die Äpfel karamellisieren.
  Auf den Palatschinken verteilen, aufrollen und warm servieren.

  \freeform{}\hrulefill{}

  \freeform{}\textit{Variante Orangen-Sahne-Füllung.}

  \ing[250]{g}{Schlagsahne}
  \ing[3]{EL}{Puderzucker}
  \ing[4]{cl}{Rum}
  \ing[2]{}{Orangen}

  Schlagsahne mit Puderzucker und Rum steif schlagen.
  Die Filets von 2 Orangen unterheben und die Masse in die Palatschinken rollen.

  \freeform{}\hrulefill{}

  \freeform{}\textit{Variante Vanilleeis und Mandeln.}

  \ing[100]{g}{Mandelblättchen}
  \ing[]{}{Vanilleeis}
  \ing[]{}{Schokosauce}

  Mandelblättchen in einer beschichteten Pfanne rösten.
  Eine Kugel Vanilleeis auf den Palatschinken setzen, zusammenklappen und mit Mandelblättchen und Schokosauce servieren.

  \freeform{}\hrulefill{}

\end{recipe}
\end{document}
