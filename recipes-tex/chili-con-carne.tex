\documentclass[../recipe-collections/cooking.tex]{subfiles}

\begin{document}
\begin{recipe}{Chili con carne} {4-5kg \faIcon{utensils}} {\faIcon{user-clock}, \faIcon{user-cog}}

  \freeform{}\textit{Michis Chili con carne.}

  \ing[2]{x}{Zwiebel, mittelgroß}
  \ing[3]{x}{Knoblauch}

  Die Zwiebeln und den Knoblauch mit Butterschmalz oder Öl im Topf anbraten.
  Die angebratenen Zwiebeln und Knoblauch aus dem Topf nehmen und beiseite stellen.

  \ing[1]{kg}{Rinderhack}
  \ing[6]{Dosen}{Tomaten passiert}

  Das Hackfleisch mit Öl anbraten.
  Fleisch anbrennen lassen, aber regelmäßig abkratzen.
  Das Fleisch darf dabei nicht verbrennen.
  Wenn der Fleischrückstand anfängt schwarz zu werden, mit den Tomaten ablöschen und dann alles vom Boden wegkratzen.

  \ing[]{}{Salz}
  Nun das Fleisch leicht salzen und die angebratenen Zwiebeln und Knoblauch wieder hinzugeben.

  \ing[3]{Dosen}{Kidneybohnen (400g)}
  \ing[2]{Dosen}{Mais (200g)}

  Nun den Mais und die Bohnen hinzugeben.

  \ing[4]{TL}{Chili con carne Gewürz}

  Abschließend das Chili con carne Gewürz hinzugeben und abschmecken.

  \freeform{}\hrulefill{}

\end{recipe}
\end{document}
