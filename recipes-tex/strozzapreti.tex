\documentclass[../recipe-collections/cooking.tex]{subfiles}

\begin{document}
\begin{recipe}{Strozzapreti}{4 Portionen \faIcon{utensils}}{20min \faIcon{user-clock}, 10min \faIcon{user-cog}}

  \freeform{}\textit{Pasta mit Linsenragout.}

  \ing[1]{x}{Karotte}
  \ing[1]{x}{Zwiebel}
  \ing[1]{EL}{Olivenöl}
  \ing[200]{g}{Berglinsen}
  \ing[4]{TL}{Bolognese Gewürz}
  \ing[100]{ml}{Wasser}
  \ing[400]{g}{passierte Tomaten}

  Die Karotte schälen und in ca. 1x1cm kleine Würfel schneiden.
  Die Zwiebel schälen, halbieren und fein würfeln.
  Das Olivenöl in einer Pfanne erhitzen und die Karotte, sowie die Zwiebel, 5 Minuten bei leichter Hitze anbraten.
  Die Berglinsen und das Bolognese Gewürz hinzufügen und 1 Minute mitbraten.
  Mit Wasser und passierten Tomaten ablöschen und ca. 15-20min köcheln lassen, bis die Linsen bissfest gegart sind.

  \ing[300]{g}{Strozzapreti}

  In einem Topf reichlich Salzwasser zum Kochen bringen, die Strozzapreti nach Packungsanleitung al dente kochen, abgießen und dabei 100ml Nudelwasser auffangen.

  \ing[250]{g}{Kirschtomaten}
  \ing[1]{Pr.}{Salz}
  \ing[1]{Pr.}{Zucker}
  \ing[100]{ml}{Nudelwasser}

  Die Kirschtomaten waschen, halbieren, zur Linsensauce geben und mit 2 Prisen Salz und Zucker würzen.
  Die gekochten Nudeln und das Nudelwasser hinzufügen und 2 Minuten bei leichter Hitze köcheln lassen.
  Die Pasta auf Tellern verteilen und sofort servieren.

  \freeform{}\hrulefill{}

  \freeform{}\faIcon{lightbulb}
  Das Rezept passt auch zu Penne, Rigatoni oder Maccaroni.

  \freeform{}\faIcon{info}
  Nährwerte pro Portion: 563kcal, 95g Kohlenhydrate, 5,4g Fett, 27g Eiweiß, 1,7g Salz

\end{recipe}
\end{document}
