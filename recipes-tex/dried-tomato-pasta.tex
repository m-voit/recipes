\documentclass[../recipe-collections/cooking.tex]{subfiles}

\begin{document}
\begin{recipe}{Pasta mit getrockneten Tomaten}{4 Portionen \faIcon{utensils}}{20min \faIcon{user-cog}}

  \freeform{}\textit{Nudelpfanne mit getrockneten Tomaten und Rucola.}

  \ing[500]{g}{Penne}
  \ing[2]{Gläser}{Tomaten getrocknet}
  \ing[1]{EL}{Olivenöl}

  Während die Nudeln kochen 2 Gläser getrocknete Tomaten in einem Sieb abtropfen lassen.
  Die getrockneten Tomaten mit etwas Olivenöl in einer großen Pfanne kurz anbraten.

  \ing[200]{g}{Rucola}

  Anschließend den Rucola in die Pfanne hineingeben.
  Sobald der Rucola nach kurzem Umrühren eingeschrumpft ist, die gekochten Penne in die Pfanne geben.

  \ing[1]{Prise}{Salz}
  \ing[1]{Prise}{Pfeffer}

  Danach mit Salz und Pfeffer abschmecken und schwenken bis die Nudeln knusprig sind.
  Als Pfeffer eignet sich Zitronenpfeffer oder schwarzer Pfeffer.

  \ing[1]{}{Parmesan}

  Am Schluss mit geriebenem Grana Padano servieren.

  \freeform{}\hrulefill{}

  \freeform{}\faIcon{burn}
  Hitzestufe 5/9 zum Erhitzen und Hitzestufe 3/9 zum Warmhalten.

\end{recipe}
\end{document}
