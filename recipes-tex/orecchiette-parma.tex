\documentclass[../recipe-collections/cooking.tex]{subfiles}

\begin{document}
\begin{recipe}{Orecchiette Parma}{4 Portionen \faIcon{utensils}}{25min \faIcon{user-cog}}

  \freeform{}\textit{Pasta mit Erbsen, Parmaschinken und Parmesan.}

  \newstep{}Den Backofen auf 160° Ober-/Unterhitze vorheizen.

  \ing[250]{g}{Erbsen, tiefgekühlt}
  \ing[150]{g}{Parmaschinken, in Scheiben}

  \textbf{15min}
  Die Erbsen aus der Kühltruhe nehmen und in einer Schüssel bis zur weiteren Verwendung beiseitestellen.
  100g Parmaschinken auf ein mit Backpapier ausgelegtes Backblech ausbreiten und mit einem weiteren Backblech abdecken.
  Anschließend den Parmaschinken 15 Minuten im Ofen knusprig backen, herausnehmen und auskühlen lassen.

  \ing[1]{x}{Zwiebel, rot}
  \ing[2]{x}{Knoblauchzehen}

  Währenddessen den restlichen Parmaschinken in feine Würfel schneiden.
  Die Zwiebel und den Knoblauch schälen und fein würfeln.

  \ing[300]{g}{Orecchiette}
  \ing[350]{ml}{Nudelwasser}

  In einem Topf reichlich Salzwasser zum Kochen bringen, die Orecchiette nach Packungsanleitung al dente kochen, abgießen und dabei 350ml Nudelwasser auffangen.

  \ing[3]{EL}{Olivenöl}
  \ing[100]{g}{Parmesan, gerieben}
  \ing[1]{Pr.}{Salz}
  \ing[2]{Pr.}{Pfeffer}
  \ing[2]{TL}{Pasta Gewürz}

  \textbf{10min}
  Das Olivenöl in einer Pfanne erhitzen, Zwiebel, Knoblauch und Schinkenwürfel in die Pfanne geben und 5 Minuten bei mittlerer Hitze anbraten.
  Die Erbsen dazugeben und 2 Minuten mitbraten.
  Die Nudeln und das Nudelwasser hinzufügen, mit dem Pasta Gewürz, 1 Prise Salz und 2 Prisen Pfeffer würzen und 2 Minuten mitköcheln lassen.
  Die Hälfte des Parmesans dazugeben und einrühren.

  \newstep{}Die Orecchiette auf Tellern verteilen, den restlichen Parmesan darüberstreuen und mit den gerösteten Schinkenchips servieren.

  \freeform{}\hrulefill{}

  \freeform{}\faIcon{info}
  Nährwerte pro Portion: 621kcal, 64g Kohlenhydrate, 27g Fett, 28g Eiweiß, 3,2g Salz

\end{recipe}
\end{document}
