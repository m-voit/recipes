\documentclass[../recipe-collections/cooking.tex]{subfiles}

\begin{document}
\begin{recipe}{Ciabatta}{2 Stück \faIcon{utensils}}{24h \faIcon{user-clock}, 30min \faIcon{user-cog}}

  \freeform{}\textit{Beilage.}

  \ing[500]{g}{Weizenmehl Typ 405}
  \ing[380]{g}{Wasser}
  \ing[4]{g}{Frischhefe}

  \ing[150]{ml}{Weinessig}
  \ing[350]{ml}{Mineralwasser, viel Kohlensäure}

  \textbf{5min}
  Das Mehl in eine Rührschüssel geben und eine kleine Mulde formen.
  Zunächst ca. 30g Wasser in die Mulde geben und die zerbröselte Hefe ins Wasser geben.
  Danach ca. 2–3 Minuten warten bis sich die Hefe leicht aufgelöst hat.
  Nun langsam umrühren bis die Menge etwas fest wird.

  \ing[15]{g}{Salz}
  \ing[15]{g}{Olivenöl}

  \textbf{20min}
  Mit der Maschine auf langsamer Stufe zu rühren beginnen und dabei ganz langsam Wasser dazugeben.
  Dabei soll der Teig nach und nach bröselig werden.
  Zunächst 8min langsam rühren, dann 6min auf \fr34 Stufe schnell rühren und dabei langsam das Salz hineingeben.
  Dann nochmal schnell rühren und dabei das Öl unterrühren.
  Am Ende soll es einen klebrigen Teig ergeben.

  \newstep{}\textbf{5min}
  Insgesamt 2h bei Raumtemperatur stehen lassen und nach jeweils 40min den Teig dehnen und falten.
  Dieser Schritt ist sehr wichtig und muss dreimal durchgeführt werden.

  \newstep{}\textbf{24h}
  Nun eine Schüssel gut mit Olivenöl einölen und den Teig 24h in einer luftdicht verschlossenen Schüssel in den Kühlschrank stellen.

  \newstep{}\textbf{30min}
  Auf einen Backschießer dick Mehl und Grieß oder Semola streuen und den Teig darauf schütten.
  Den Teig dabei nicht drücken.
  Anschließend den Teig in zwei gleich große Stücke schneiden, in Form schieben, dick Mehl darüber streuen und mit einem Tuch abgedeckt 30min bei Raumtemperatur akklimatisieren lassen.
  Auch hier darauf achten, dass der Teig nicht gedrückt und sanft in Form geschoben wird, damit die Luft im Teig bleibt.

  \newstep{}\textbf{15min}
  Den Backofen mit Backstein auf 250~°C vorheizen.
  Zusätzlich ein Backblech oder Steine auf die unterste Schiene legen und ebenfalls stark erhitzen.
  Die Ciabatta vorsichtig vom Backschießer auf den heißen Backstein setzen und die Ofentür sofort schließen.
  30–60~Sekunden backen.
  Anschließend ca. 150~ml Wasser auf die heißen Steine oder das Blech gießen, sodass es sofort verdampft (Schwaden erzeugen).
  Die Ofentür direkt wieder schließen und 15~Minuten backen.

  \newstep{}\textbf{10min}
  Nun die Ofentür für 30~Sekunden öffnen, um den Dampf abzulassen, dann weitere 5~Minuten fertigbacken.
  Den Ofen ausschalten, die Tür öffnen und die Ciabatta noch 5~Minuten im Ofen ruhen lassen.
  Anschließend auf einem Gitter vollständig auskühlen lassen.
  Die gesamte Backzeit beträgt 25~Minuten.

  \newstep{}\textbf{24h}
  Für die Teigruhe im Kühlschrank eine große Schüssel verwenden, da der Teig sein Volumen etwa verdreifacht bis vervierfacht.

  \freeform{}\hrulefill{}

\end{recipe}
\end{document}
