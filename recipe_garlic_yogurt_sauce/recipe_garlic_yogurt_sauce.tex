\documentclass[
  DIV=11,% DIV Faktor für Satzspiegelberechnung - muss bei anderen Schriftgrößen als 11pt angepasst werden , sie Doku zu KOMA Script
  pagesize,% write pagesize to DVI or PDF
  fontsize=11pt,% use this font size
  paper=a4,% use ISO A4
]{scrartcl}

\usepackage[utf8]{inputenc}
\usepackage[sfdefault, scaled=.95, book]{FiraSans}
\usepackage[T1]{fontenc}
\usepackage{textcomp}
\usepackage{graphicx}
\usepackage{caption}
\usepackage{subcaption}
\usepackage[ngerman]{babel}
\usepackage{microtype}% optischer Randausgleich etc.
\usepackage[nonumber]{cuisine}
\usepackage{nicefrac}

\selectlanguage{ngerman}
\pagenumbering{gobble}

\begin{document}
\begin{recipe}{Knoblauch-Joghurt-Sauce} {250ml} {30 Minuten}

\freeform
\textit{Ergibt eine Joghurt-Sauce mit Knoblauch Aromen.}

\ingredient[150]{g}{Joghurt mild}
\ingredient[100]{g}{Crème Fraîche}
\ingredient[1]{EL}{Mayonnaise}

Zuerst Joghurt, Crème Fraîche und Mayonnaise verrühren.

\ingredient[2]{x}{Knoblauchzehen}
\ingredient[1]{EL}{Dill}
\ingredient[2]{EL}{Schnittlauch}
\ingredient[1]{EL}{Zitronensaft}

Knoblauchzehen schälen, hacken und unterrühren.
Dill, Schnittlauch und Zitronensaft untermischen.

\ingredient[]{}{Pfeffer schwarz}
\ingredient[]{}{Meersalz}

Mit Salz und Pfeffer abschmecken.
In eine saubere Flasche füllen und anschließend über Nacht ziehen lassen.

\freeform
\hrulefill

\freeform 
\textbf{Haltbarkeit:}
Hält sich im Kühlschrank eine Woche.

\end{recipe}
\end{document}
