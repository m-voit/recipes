\documentclass[
  DIV=11,% DIV Faktor für Satzspiegelberechnung - muss bei anderen Schriftgrößen als 11pt angepasst werden , sie Doku zu KOMA Script
  pagesize,% write pagesize to DVI or PDF
  fontsize=11pt,% use this font size
  paper=a4,% use ISO A4
]{scrartcl}

\usepackage[utf8]{inputenc}
\usepackage[sfdefault, scaled=.95, book]{FiraSans}
\usepackage[T1]{fontenc}
\usepackage{textcomp}
\usepackage{graphicx}
\usepackage{caption}
\usepackage{subcaption}
\usepackage[ngerman]{babel}
\usepackage{microtype}% optischer Randausgleich etc.
\usepackage[nonumber]{cuisine}
\usepackage{nicefrac}

\selectlanguage{ngerman}
\pagenumbering{gobble}

\begin{document}
\begin{recipe}{BBQ-Sauce} {750ml} {60 Minuten}

\freeform
\textit{Ergibt eine würzige und leicht scharfe BBQ-Sauce.}

\ingredient[1]{x}{Zwiebel}
\ingredient[2]{x}{Knoblauchzehe}
\ingredient[1]{x}{Apfel}
\ingredient[300]{g}{Tomaten}
\ingredient[1]{EL}{Rapsöl}

Zuerst Zwiebel, Knoblauchzehen und Apfel schälen und fein würfeln.
Tomaten würfeln.
Rapsöl in einer beschichteten Pfanne erhitzen und die Zwiebel darin anbraten.

\ingredient[100]{g}{Tomatenmark}
\ingredient[125]{g}{Rohrzucker}
\ingredient[150]{ml}{Apfelessig}
\ingredient[50]{ml}{Worcestersauce}
\ingredient[50]{ml}{Whisky rauchig}
\ingredient[\fr12]{TL}{Chilipulver}
\ingredient[2]{TL}{Paprikapulver rosenscharf}
\ingredient[\fr12]{TL}{Pfeffer schwarz}
\ingredient[1]{TL}{Meersalz}

Nach ca. 1 Minute Knoblauch, Apfel und Tomaten zufügen.
Nach weiteren 2 Minuten Tomatenmark, Rohrzucker, Apfelessig, Worcestersauce und Whisky zufügen und gut verrühren.
Dann 1/2 TL Chilipulver, 2 TL Paprikapulver, 1/2 TL Pfeffer und 1 TL Meersalz zugeben.

\newstep
Die Sauce ca. 30-40 Min. bei schwacher Hitze köcheln lassen.
Sauce pürieren und sofort in eine saubere Flasche füllen.
Anschließend über Nacht ziehen lassen.

\freeform
\hrulefill

\freeform
\textbf{Hitzestufe:}
Hitzestufe 7 zum Anbraten und Hitzestufe 4-5 zum köcheln lassen.

\freeform 
\textbf{Haltbarkeit:}
Hält sich im Kühlschrank mehrere Wochen.

\freeform 
\textbf{Tipp:}
Geschmack kann durch die Verwendung von unterschiedlichen Whiskys variiert werden.

\end{recipe}
\end{document}
