\documentclass[
  DIV=11,% DIV Faktor für Satzspiegelberechnung - muss bei anderen Schriftgrößen als 11pt angepasst werden , sie Doku zu KOMA Script
  pagesize,% write pagesize to DVI or PDF
  fontsize=11pt,% use this font size
  paper=a4,% use ISO A4
]{scrartcl}

\usepackage[utf8]{inputenc}
\usepackage[sfdefault, scaled=.95, book]{FiraSans}
\usepackage[T1]{fontenc}
\usepackage{textcomp}
\usepackage{graphicx}
\usepackage{caption}
\usepackage{subcaption}
\usepackage[ngerman]{babel}
\usepackage{microtype}% optischer Randausgleich etc.
\usepackage[nonumber]{cuisine}
\usepackage{nicefrac}

\selectlanguage{ngerman}
\pagenumbering{gobble}

\begin{document}
\begin{recipe}{BBQ-Sauce mit Kellerbier} {600ml} {35 Minuten}

\freeform
\textit{Ergibt eine scharf würzige BBQ-Sauce mit malziger Bier Note, die zu gegrilltem Fleisch aller Art passt.}

\ingredient[340]{ml}{Kellerbier süffig}
\ingredient[400]{ml}{Tomaten passiert}
\ingredient[8]{EL}{Honig}
\ingredient[4]{EL}{Olivenöl}
\ingredient[4]{EL}{Worcester Sauce}
\ingredient[2]{EL}{Tomatenmark}
\ingredient[2]{TL}{Dijon-Senf}
\ingredient[1]{x}{Zwiebel rot, mittelgroß}
\ingredient[2]{x}{Knoblauchzehe}
\ingredient[1]{TL}{Salz}
\ingredient[1]{TL}{Pfeffer}
\ingredient[\fr12]{TL}{Paprikapulver geräuchert}
\ingredient[\fr12]{TL}{Chipotle Chili}

Zunächst Zwiebel abziehen und in feine Würfel schneiden.
Den Knoblauch ebenfalls abziehen und fein hacken.
Das Öl in einem Topf erhitzen und darin die Zwiebeln und den Knoblauch glasig anschwitzen.
Das Ganze nun mit dem Kellerbier ablöschen.
Die passierten Tomaten, den Senf, den Honig, die Worcester Sauce, das Tomatenmark, den Pfeffer, die Chipotle Chili, die geräucherte Paprika und das Salz einrühren.
Anschließend kurz aufkochen lassen.

\newstep
Bei geringer Hitze die Sauce etwa 20 Minuten köcheln lassen.
Zum Abschluss die Sauce abschmecken und ggf. nachwürzen.
Anschließend über Nacht ziehen lassen.

\freeform
\hrulefill

\freeform 
\textbf{Hitzestufe:}
Hitzestufe 7 zum Anbraten und Hitzestufe 4-5 zum köcheln lassen.

\freeform 
\textbf{Haltbarkeit:}
Hält sich im Kühlschrank mehrere Wochen.

\freeform 
\textbf{Tipp:}
Zur Verwendung als Glasur für Grillgut die Soße weniger lange einkochen lassen, sodass sie eine dünnflüssige Konsistenz behält.

\freeform 
\textbf{Modifikationen:}
6 EL Honig verwenden und Schärfe reduzieren.

\end{recipe}
\end{document}
