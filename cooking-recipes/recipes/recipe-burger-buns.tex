\begin{recipe}{Burger Buns} {4 Portionen} {30 Minuten Arbeitszeit, 120 Minuten Wartezeit, 150 Minuten Gesamt}

  \freeform
  \textit{}

  \ingredient[240]{g}{Weizenmehl Typ 550}
  \ingredient[16]{g}{Hefe frisch}
  \ingredient[20]{g}{Zucker}
  \ingredient[1]{TL}{Salz}
  \ingredient[90]{ml}{Milch}
  \ingredient[2]{}{Ei}
  \ingredient[12,5]{g}{Butter}

  Hefe und Zucker vermengen, damit sich die Hefe sich auflöst.
  Die restlichen Zutaten (ohne Eiweiß) in die Schüssel deiner Küchenmaschine hineingeben.
  Sobald die Hefe flüssig ist, schüttest du sie ebenfalls in die Rührschüssel.
  Jetzt den Teig für ca. 15 Minuten in der Maschine durchkneten und anschließend an einem warmen Ort für 60 Minuten gehen lassen. (Volumen verdoppelt sich)

  \newstep
  Nach der ersten Phase nimmst du den Teig aus der Schüssel und machst dir ca. 90g Portionen.
  Die einzelnen Teiglinge werden nun geschliffen.
  Letztlich sind es nur Kugeln in die der Teig nach innen eingearbeitet wurde und so eine Oberflächenspannung entsteht, die sie später gut aufgehen lassen.
  Deine fertigen Teiglinge setzt du jetzt auf ein bemehltes Backblech und lässt sie weitere 45 Minuten im 40 Grad warmen Backofen gehen.
  In der Zwischenzeit kannst du schon mal das übrig gebliebenen Eiweiß und und das Wasser kurz verquirlen.

  \newstep
  Wenn die Teiglinge gut aufgegangen sind, holst du sie aus dem Ofen und heizt diesen auf 205 Grad Ober-/Unterhitze vor.
  Wenn die Temperatur erreicht ist, die Brötchen noch kurz mit der Eiweißmischung einpinseln und dann für 10 Minuten im Ofen backen.
  Achtung ab Min 8 am Ofen bleiben, die werden sehr schnell dunkel! Nach dem Backen auf einem Abkühlgitter auskühlen lassen.

  \freeform
  \hrulefill

  \freeform
  \textbf{Hinweise:}
  Gut bemehltes Arbeitsbrett verwenden. Eigelb in den Teig, Eiweiß zum Einpinseln. Backpapier verwenden.

\end{recipe}
