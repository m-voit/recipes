\begin{recipe}{Chili-Chutney} {1 Glas} {15 Minuten Arbeitszeit, 120 Minuten Wartezeit, 135 Minuten Gesamt}

  \freeform
  \textit{Beilage zu Fleischgerichten.}

  \ingredient[30]{}{rote Chilis (mild bis mittelscharf)}
  \ingredient[230]{ml}{Tafelessig}
  \ingredient[1]{Stück}{frischer Ingwer (ca. 1–2 cm)}
  \ingredient[1]{TL}{Öl}
  \ingredient[120]{g}{Zucker}
  \ingredient[5]{EL}{Wasser}
  \ingredient[1]{TL}{Kreuzkümmel}
  \ingredient[1]{Prise}{Salz}

  Chilis waschen, die Kerne entfernen und das Fruchtfleisch in kleine Würfel schneiden.
  Würfel mit dem Tafelessig mischen und darin ca. 2 Std. einweichen.

  \newstep
  Ingwer schälen und in feine Würfel hacken.
  Öl in einer Pfanne erhitzen und den Ingwer darin ca. 3 Min. dünsten.
  Ingwer zu den Chilis geben.

  \newstep
  Zucker mit dem Wasser in eine Pfanne geben und den Zucker unter Rühren im Wasser auflösen.
  Dann die Chilis, Kreuzkümmel und Salz hinzugeben und das Ganze bei mittlerer Hitze ca. 5 Minuten köcheln lassen, bis das Chutney eindickt.

  \newstep
  Chutney entweder sofort in ein sterilisiertes Glas füllen oder etwas abkühlen lassen und direkt genießen.

  \freeform
  \hrulefill

  \freeform
  \textbf{Tipp:}
  Einmachglas sterilisieren.

  \end{recipe}
