\documentclass[
  DIV=11,% DIV Faktor für Satzspiegelberechnung - muss bei anderen Schriftgrößen als 11pt angepasst werden , sie Doku zu KOMA Script
  pagesize,% write pagesize to DVI or PDF
  fontsize=11pt,% use this font size
  paper=a4,% use ISO A4
]{scrartcl}

\usepackage[utf8]{inputenc}
\usepackage[sfdefault, scaled=.95, book]{FiraSans}
\usepackage[T1]{fontenc}
\usepackage{textcomp}
\usepackage{graphicx}
\usepackage{caption}
\usepackage{subcaption}
\usepackage[ngerman]{babel}
\usepackage{microtype}% optischer Randausgleich etc.
\usepackage[nonumber]{cuisine}
\usepackage{nicefrac}

\selectlanguage{ngerman}
\pagenumbering{gobble}

\begin{document}
\begin{recipe}{Feurige Salsa Sauce} {750ml} {50 Minuten}

\freeform
\textit{Ergibt eine feurige Sauce.}

\ingredient[1]{x}{Zwiebel}
\ingredient[2]{x}{Knoblauchzehe}
\ingredient[2]{x}{Chilischoten (Jalapenos)}
\ingredient[1]{EL}{Öl}

Zuerst Zwiebel und Knoblauchzehen schälen und fein würfeln.
Kerne der Chilischoten entfernen und Schoten würfeln.
Öl in einer beschichteten Pfanne erhitzen und die Zwiebeln, sowie den Knoblauch glasig dünsten.

\ingredient[425]{ml}{Stückige Tomaten}
\ingredient[4]{EL}{Apfelessig}
\ingredient[2]{EL}{Rohrzucker braun}

Anschließend Chilischoten, stückige Tomaten, Apfelessig und Rohrzucker in die Pfanne geben.

\ingredient[]{}{Pfeffer schwarz}
\ingredient[]{}{Meersalz}

Die Sauce ca. 30-40 Min. bei schwacher Hitze köcheln lassen.
Gelegentlich umrühren und mit Salz und Pfeffer, sowie Apfelessig abschmecken.
Heiß in eine saubere Flasche füllen und anschließend über Nacht ziehen lassen.

\freeform
\hrulefill

\freeform
\textbf{Hitzestufe:}
Hitzestufe 7 zum Anbraten und Hitzestufe 4-5 zum köcheln lassen.

\freeform 
\textbf{Haltbarkeit:}
Hält sich im Kühlschrank mehrere Wochen.

\freeform 
\textbf{Tipp:}
Geschmack kann durch die Verwendung von unterschiedlichen Chilis variiert werden.

\end{recipe}
\end{document}
