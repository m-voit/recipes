\documentclass[
  DIV=11,% DIV Faktor für Satzspiegelberechnung - muss bei anderen Schriftgrößen als 11pt angepasst werden , sie Doku zu KOMA Script
  pagesize,% write pagesize to DVI or PDF
  fontsize=11pt,% use this font size
  paper=a4,% use ISO A4
]{scrartcl}

\usepackage[utf8]{inputenc}
\usepackage{makeidx}
\usepackage{amsfonts}
% \usepackage[slantedGreek,sc]{mathpazo}% Schriftart Palatino
\usepackage{lmodern}% statt mathpazo, falls CM fonts verwendet werden sollen
\usepackage[scaled=.95]{helvet}
\usepackage{courier}
\usepackage[T1]{fontenc}
\usepackage{textcomp}
\usepackage{amsmath}% standard math notation (vectors/sets/...)
\usepackage{bm}% standard math notation (fonts)
\usepackage{fixmath}% standard math notation (fonts)
\usepackage{graphicx}
\usepackage{caption}
\usepackage{subcaption}
\usepackage{scrlayer-scrpage}
\usepackage[ngerman]{babel}
\usepackage{ellipsis}% Korrigiert den Weißraum um Auslassungspunkte
\usepackage{microtype}% optischer Randausgleich etc.
\usepackage{cuisine}[nonumber]
\usepackage{nicefrac}

\selectlanguage{ngerman}
\pagenumbering{gobble}

\begin{document}
\begin{recipe}{Sour Cream} {4 Personen} {10 Minuten}

\freeform
\textit{Ideal als Dip zu Ofenkartoffeln.}

\ingredient[150]{ml}{Schmand}
\ingredient[150]{ml}{Saure Sahne}
\ingredient[2]{TL}{Zwiebel gehackt}
\ingredient[2]{x}{Knoblauchzehe}
\ingredient[1]{Bund}{Schnittlauch}
\ingredient[]{}{Salz}
\ingredient[]{}{Pfeffer}

Für die Zubereitung der Sour Cream alle Zutaten in einer Schale vermengen und später mit Salz und Pfeffer abschmecken.
Danach die Sour Cream im Kühlschrank ziehen lassen.

\freeform
\hrulefill

\freeform 
\textbf{Tipp:}
Über Nacht ziehen lassen um den Dip milder werden zu lassen.

\end{recipe}
\end{document}
